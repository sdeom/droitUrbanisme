\documentclass[10pt,a4paper,twoside]{book}

\usepackage{makeidx}
\usepackage[utf8]{inputenc}
\usepackage[T1]{fontenc}
\usepackage{graphicx}
\usepackage{eurosym}
\usepackage{enumitem}
\usepackage{xspace}
\usepackage{moredefs}\usepackage[mla]{lips}
% à passer en dernier
\usepackage[french]{babel,varioref}
\usepackage{hyperref}

\author{Me Bernard \nom{Larmorlette}}
\title{Droit de l'Urbanisme et de l'Aménagement}
\date{Février 2020}

\hypersetup{%
%	pdfinfo={%
%		Title={Droit de l'Urbanisme et de l'Aménagement}%
%		, Subject={}%Marché \@ReferenceMarche}%
%		, Author={Samuel Déom}%
%		%, Keywords={}%\@ReferenceMarche}%
%	}%
	, colorlinks = true% colore, plutot qu'encadre, les liens hypertexte
	, linkcolor = black% colore les liens internes en noir
	, urlcolor = black% colore les liens externes en noir
	, breaklinks = true% autorise les liens à être étendus sur plusieurs lignes
}

% Ecrire du texte juridique
\makeatletter
\newcommand*{\assPlen}{\@ifstar{\mbox{ass. plén.}\xspace}{assemblée plénière\xspace}}
\newcommand*{\civUn}{\@ifstar{civ. 1\iere{}\xspace}{première chambre civile\xspace}}
\newcommand*{\civDeux}{\@ifstar{civ. 2\ieme{}\xspace}{deuxième chambre civile\xspace}}
\newcommand*{\civTrois}{\@ifstar{civ. 3\ieme{}\xspace}{troisième chambre civile\xspace}}
\newcommand*{\CourDeCas}{\@ifstar{Cass.\xspace}{Cour de Cassation\xspace}}
\newcommand*{\CA}{\@ifstar{CA\xspace}{Cour d'Appel\xspace}}
\newcommand*{\CE}{\@ifstar{CE\xspace}{Conseil d'État\xspace}}
\makeatother

\newcommand*{\jurisCourDeCas}[3][]{\CourDeCas* #2, \printdate{#3}\ifthenelse{\equal{#1}{}}{}{, \no#1}}% Numéro de chambre, date et en option no de pourvoi
\newcommand*{\jurisCA}[2]{\CA* #1, \printdate{#2}}%Ville, date

\newcommand*{\refArticle}[2]{\mbox{#1.\,#2}}
\newcommand*{\articleCodifie}[2]{\mbox{article~\refArticle{#1}{#2}}}
\newcommand*{\articlesCodifies}[2]{\mbox{articles~\refArticle{#1}{#2}}}
\newcommand*{\ArticleCodifie}[2]{\mbox{Article~\refArticle{#1}{#2}}}
\newcommand*{\ArticlesCodifies}[2]{\mbox{Articles~\refArticle{#1}{#2}}}
\newcommand*{\articlesCodifiesEtSuivants}[2]{\articlesCodifies{#1}{#2} et suivants}
\newcommand*{\articleDu}[3][]{\ifthenelse{\equal{#1}{}}{\mbox{article~#2}}{\articleCodifie{#1}{#2}} du #3}
\newcommand*{\articlesDu}[3][]{\ifthenelse{\equal{#1}{}}{\mbox{articles~#2}}{\articlesCodifies{#1}{#2}} du #3}
\newcommand*{\articlesDuEtSuivants}[3][]{\ifthenelse{\equal{#1}{}}{\mbox{articles~#2}}{\articlesCodifies{#1}{#2}} et suivants du #3}
\newcommand*{\ArticleDu}[3][]{\ifthenelse{\equal{#1}{}}{\mbox{Article~#2}}{\ArticleCodifie{#1}{#2}} du #3}
\newcommand*{\ArticlesDu}[3][]{\ifthenelse{\equal{#1}{}}{\mbox{Articles~#2}}{\ArticlesCodifies{#1}{#2}} du #3}
\newcommand*{\ArticlesDuEtSuivants}[3][]{\ifthenelse{\equal{#1}{}}{\mbox{Articles~#2}}{\ArticlesCodifies{#1}{#2}} et suivants du #3}

\makeatletter
\newcommand*{\ca}{\@ifstar{\mbox{C.~assur.}\xspace}{Code des assurances\xspace}}
\newcommand*{\cch}{\@ifstar{CCH\xspace}{Code de la construction et de l'habitation\xspace}}
\newcommand*{\cciv}{\@ifstar{\mbox{C.~civ.}\xspace}{Code civil\xspace}}
\newcommand*{\ccom}{\@ifstar{\mbox{	C.~com.}\xspace}{Code du commerce\xspace}}
\newcommand*{\ccons}{\@ifstar{\mbox{C.~consom.}\xspace}{Code de la consommation\xspace}}
\newcommand*{\ccp}{\@ifstar{CCP\xspace}{Code de la commande publique\xspace}}
\newcommand*{\cpc}{\@ifstar{CPC\xspace}{Code de procédure civile\xspace}}
\newcommand*{\cpi}{\@ifstar{CPI\xspace}{Code de la propriété intellectuelle\xspace}}
\newcommand*{\cu}{\@ifstar{\mbox{C.~urb.}\xspace}{Code de l'urbanisme\xspace}}
\makeatother
\newcommand*{\articleCodifie}[2]{article~#1.\,#2}
\newcommand*{\articlesCodifies}[2]{articles~#1.\,#2}

% Conservé pour compatibilité avec première version
\newcommand*{\articleCU}[2]{\articleDu[#1]{#2}{\cu}\index{#1#2@#1.\,#2}}
\newcommand*{\articlesCodifiesS}[2]{articles~#1.\,#2 et suivants}

% Commandes utiles souvent
\newcommand*{\cad}{c'est-à-dire\xspace}
\newcommand*{\commune}{collectivité locale\xspace}
\newcommand*{\CM}{Conseil Municipal\xspace}
\newcommand*{\CU}{Code de l'urbanisme\xspace}
\newcommand*{\etc}{\emph{etc}.\xspace}
\newcommand*{\metreCarre}{m\up{2}\xspace}
\newcommand*{\montant}[1]{\nombre{#1}~\euro}
\newcommand*{\nom}[1]{\textsc{#1}}
\newcommand*{\pourcent}[1]{\nombre{#1}~\%}
\newcommand*{\prixSurface}[1]{\montant{#1}/m\up{2}}
\newcommand*{\PUP}{projet ubain partenarial\xspace}
\newcommand*{\sdp}{surface de plancher\xspace}

% Pour faciliter l'intégration denote de Magalie ou de Julie
\newcommand*{\CAD}{\cad}


\begin{document}

	\maketitle

	\chapter[Introduction]{Introduction générale au droit de l'urbanisme et de l'aménagement}

	\section{Une police administrative spécialisée}
		
		Le droit de l’urbanisme n’est pas un droit contractuel, mais un \textbf{droit de police administrative}.
		
		Cela a plusieurs conséquences : 

		-	Il ne se négocie pas. Nous avons quelques thématiques ou sous thématique en urbanisme ou aménagement l’utilisation de contrat. Mais généralement c’est du droit unilatéral CAD du droit de police

		On ne peut pas aller négocier en conseil municipal ou avant un zonage du PLU

		Pas de négociation du permis : ça ne veut pas dire ne pas échanger avec le service instructeur, communal ou intercommunal, pour que l’architecte ou le MO comprenne quel est la lecture de l’instructeur de tel ou tel prescription du PLU ou du CDU. L’échange informel est possible. Cela est acquis. En revanche à partir du moment où le dépôt officiel du PC a été réalisé, alors le processus administratif d’instruction est lancé avec des délais et une procédure réglementée.

		On ne peut pas non plus sur une parcelle donnée, aller plaider ma cause si je dépasse par exemple la hauteur, cela ne se négocie pas. Les prescriptions doivent être respectées en conformité.

		Je ne peux pas négocier le permis avec le service instructeur suite aux instructions particulières comme le SDIS (service départementale incendie sécurité)

		C’est un processus unilatéral qui ne se contractualise pas

		Si un recours est introduit contre le PC, le requérant n’a pas facialement la possibilité d’aller négocier avec la ville.

		Pas de négociation de zonage : dans le cadre d’un PLU je peux échanger avec les conseillers municipaux ou intercommunaux, avant le vote, en cours d’élaboration d’un PLU, mais une fois que le PLU est arrêté, voir approuvé (dernière délibération qui entérine le PLU), ce processus n’est pas négociable. Je peux juste m’exprimer en amont dans le cadre d’une concertation ou en aval au stade de l’enquête publique.

		Pas de négociation possible donc en phase de planification

		
		Il y a en revanche quelques secteurs, quelques sous matière en urba et aménagement, où la négociation est possible au travers d’un contrat : 

		
		-	ZAC : zone d’aménagement concerté, concession d’aménagement. Au sein des ZAC il y a des traités de concessions, contrat administratif entre une collectivité locale et un aménageur appelé concessionnaire, ils négocient une convention administrative appelé traité de concession qui constitue la règle du jeu contractuelle durant les années pendant lesquelles l’opération de ZAC va être mise en œuvre

		-	Le droit de la fiscalité de l’urbanisme : deux outils d’essence contractuelle : la convention de PUP (projet urbain partenarial) est un mode de financement d’équipement public souvent à partir de projet d’initiative privée, ce qui le distingue des opérations de ZAC qui sont d’initiative publique dans lesquelles, outre le traité de concession, il peut y avoir des conventions de participation (convention entre la commune concédant d’une opération d’aménagement et des propriétaires de terrains, ou les bénéficiaires de promesses consenties par ces propriétaires comme les promoteurs qui développerai en ZAC des terrains, pour lesquels il conviendrait préalablement à l’obtention du PC, de signer des conventions de participation.

		Ici on détermine le montant du quantum à verser à la ville, voir à son aménageur, par le propriétaire ou le promoteur qui en ZAC développerait le terrain

		C’est un droit de police car sanctions radicales. Exemple : 

		-	PC valant division : chacun a sur sa parcelle une partie du permis commun, chacun est co-MO. Et chacun doit faire un petit immeuble mais certaines exécutions du PC ne sont pas conformes au plan. Avant même l’achèvement de l’immeuble, cette non-conformité si elle peut être décelée, elle constitue une infraction pénale, et le maire peut prendre un arrêté interruptif de travaux.

		-	Une annexe qui est construite sur le fonds mitoyen sans PC sans autorisation : infraction pénale. Si le maire est inactif, le préfet peut intervenir et prendre un arrêté interruptif de chantier et demander la démolition de la construction immobilière

		
		(un PC fabrique une SDP surface de plancher)

		
		
	\section{Un droit décentralisé}
	
		Ce droit a une dimension intercommunale.
		
		Décentralisation ancienne, des lois de 1982 (décentralisation des CL) et 1983 (décentralisation de l’urbanisme)

		-	Décentralisation dans l’élaboration la fabrication des PLU (avant les POS)

		-	Décentralisation des instructions de constructions individuelles (autorisations administratives PC)

		En 1985, décentralisation des compétences en matières d’aménagement des ZAC, donc décentralisées au CL.

		
		Ici, on rapproche le pouvoir décisionnaire des administrés. L’état sera sous surveillance démocratique des administrés qui sera renforcée avec une sanction électorale en fin de course, par les urnes de la part des administrés, c’était la vision et le concept avec la décentralisation.

		
		Il y a un risque de collusion, de dérive

		
		Autre conséquence de la décentralisation : transfert de la responsabilité administrative, on le voit sur le contentieux de la responsabilité. La décentralisation est la prise de responsabilité

		Tout acte illégal d’une Collectivité publique, fait présumer sa responsabilité. En conséquence 3 exemples : 

		-	Maire délivre un PC après instruction. Il est attaqué et annulé par le juge administratif. Cet acte annulé est susceptible d’engager ma responsabilité de la commune par le pétitionnaire qui a engagé des honoraires d’architecte et de bureau d’étude

		-	Un refus illégal. Annulation du refus par le juge. Le coût de la construction est désormais différent, responsabilité possible de la commune sur ce préjudice

		-	Je dispose d’un terrain porte un zonage qui va réduire ou anéantir ma constructibilité. J’attaque le zonage concerné ou le PLU et je le fais annulé. Le précédent zonage redevient applicable, j’avais un acquéreur mais je ne l’ai plus, en théorie la responsabilité de la collectivité peut être recherchée

		
		Donc dans les années 80 les collectivités se sont retrouvées face à leur responsabilité. Donc question d’une éventuelle assurance. L’état n’était pas assuré mais les collectivités pour leur nouvelle compétence, certaines sont désormais assurées.

		Paris n’est pas assuré*

		
		Sur l’intercommunalité : elle est intervenue progressivement, alors qu’elle n’était pas obligatoire =. Besoin de faire face à une technicité toujours grandissante en la matière

		Donc les communes se sont regroupées de manière facultative dans un premier temps puis de manière obligatoire pour 3 blocs de compétence

		-	Pour les documents d’urbanisme comme le PLU. Il y a les communautés de communes et les communautés d’agglomérations, et des communautés urbaines comme à LYON

		Cela a des avantages pour la mutualisation des moyens, mais il y a des inconvénients. En cas de contestation du PLU sur les moyens de légalité externe, si j’arrive à mes fins cela fait tombé tout le PLU pour toutes les communes

		
		-	L’instruction des autorisations. Cela est sensible. Ce bloc va être découpé en plusieurs phases. Les maires de chaque commune veulent garder l’apposition de leur signature. Il va avoir une phase intercommunale avec l’instruction mais l’acte final sera municipal 

		La communauté de commune demande l’avis du maire sur le projet

		Après tout le processus d’instruction c’est l’intercommunauté qui l’assume

		Puis il y a une proposition faite au maire, soit acceptation du projet avec ou sans prescription technique. C’est formel mais pas opposable.

		Si la compétence de signataire reste à la mairie, le maire reste libre 

		Si tout le monde est en phase, préparation de l’arrêté

		Le Maire  ne suit pas l’interco c’est la responsabilité du maire, si suit l’avis de l’interco alors c’est l’intercommunalité qui doit prendre la responsabilité

		
	
	\section{Un droit instable}
	
		L'instabilité est due aux textes successifs et au contentieux local qui frappe les documents réglementaires.
		
		Il y a un texte transversal tous les 2 ans qui toilette code de la construction de l’environnement de la santé de l’urbanisme

		Donc toujours grosse mise à jour avec ajout de textes nouveaux. Depuis notamment la loi SRU Solidarité et renouvellement urbain 2000 : on a changé les étiquettes SCOT et POS

		Et depuis tous les 2-3 ans beaucoup de changement.

		Cela s’explique par le fait que tout est spatial, tout se réfère au document d’urba (vélo locaux commerciaux…)

		Donc instabilité nationale

		Instabilité également locale, les modifications au niveau de chaque commune et intercommunalité avec les PLU.

		Les PLU sont fait pour s’inscrire dans le temps, mais si le PLU oublie de rendre possible un grand projet il faudra le modifier (au sens large)

		Troisième instabilité : du fait de la jurisprudence, elle est très sollicitée dans cette matière car c’est la troisième matière administrative en quantité de contentieux administratif.

		Ces ombreux contentieux génèrent des positions pas toujours arbitrées de la même manière.

		Ex : Droit de préemption anti vente à la découpe. La collectivité peut préempter quand propriétaire veut vendre à la découpe son immeuble.

		La commune peut préempter pour des logements sociaux. Il y a quelques années deux préemptions pour deux appartements du même immeuble. Deux chambres du Tribunal statuent mais rendent des décisions différentes.

		Donc instabilité du droit applicable

		
	
	\section{Un droit multiple}
	
		Son caractère tient à la multiplicité des intervenant et s'exprime par des actes individuels et réglementaires.
		
		Par ses acteurs

		L’état central, avec pilotage de grande opération

		Etablissement public d’Aménagement : Ici, la ville gère son PLU et les autorisations individuelles mais la dimension aménagement est gérée par l’état.

		L’état au niveau départementale, c’est une intervention ancienne, la présence du préfet a accompagné le processus de décentralisation, avec comme corolaire donc du contrôle de légalité a postériori, c’est la manière pour l’état de garder un contrôle a posteriori pour tous les actes décentralisés.

		En 1992, il y avait des statistiques en urbanisme sur le pourcentage d’acte censuré par le préfet 0,072% de tous les actes des CT 

		Dernière émergence de l’état, c’est le préfet de région à qui il revient de recevoir les études d’impact qui accompagne soit les ZAC ou certains PC lourds au moins >10.000 m² de SDP.

		Alors avis, pour savoir si ces éléments constitutifs d’un dossier de ZAC ou de permis ne sont pas complets.

		Il y a aussi les OPF, les établissements publics fonciers. Il est parfois régional ou intercommunal. Ce sont des établissements publics avec une personnalité juridique à qui on donne un rôle de portage foncier CAD activité ou mission consistant à devenir propriétaire d’un terrain avant de lui donner une dimension ou une affectation collective.

		Ils sont interdits de mission en tant qu’aménagement. Un OPF n’est pas un aménageur. Ses ressources sont des taxes qui y sont affectés et des ressources des collectivités.

		Comme opérateur public, il y a l’architecte des bâtiments de France qui est pris dans l’instruction, puis il y a les opérateurs privés, les maîtres d’œuvre comme l’architecte, les géomètres

		Il y a les opérateurs (terme générique qui englobe promoteur lotisseur investisseur MO) souvent assisté par un AMO, un assistant à maître d’ouvrage

		Il peut avoir comme opérateur, des marchands de biens aussi. Il y a les lotisseurs, et les aménageurs. Dans les aménageurs (opération importante d’aménagement comme des ZAC et concession) il y a les aménageurs privés, les société d’économie mixte locale SEML, à l’intérieur, il y a les SEM in house capitalisée par les collectivités locales avec un avantage dans certaines ZAC, et les SEM classique avec un partage équivalent entre société privée et collectivité locale, autre cas d’aménageur c’est la collectivité locale alors c’est une opération un aménagement en régi.

		
		
		
		
		
		Par la nature de ses actes

		3 grandes catégories d’actes :

		-	Les actes individuels : PC, autorisation de lotir devenu permis d’aménager, la décision de préemption.

		-	A : PLU, SCOT schéma de cohérence territoriale, les cartes communales (mini PLU)

		-	Les contrats, ils sont rares (3 exemples vus ci-dessus)

		Dans une étude de 1991, une catégorie d’actes intermédiaires : actes non réglementaires qui comprendraient l’acte créant une ZAC, acte hybride et les DUP, les déclaration d’utilité publique qui sont des arrêtés préfectoraux qui portent expropriation de terrain

		Ils seraient non réglementaires car concernent une pluralité d’administrés non identifiés ou identifiables.

		Ex : quand je créé une ZAC j’insère un nombre identifiable propriétaire de parcelle

		
		Les actes individuels : 

		Première caractéristique : les effets par la notification de l’acte

		ils ont un processus d’opposabilité particulier, ils produisent des effets juridiques de manière distincte que les actes réglementaires

		Effet dès notification par l’administration à son pétitionnaire

		Deuxième caractéristique : les effets juridiques

		Un PC dès lors qu’il est délivré, notifié, il donne, dès réception, des droits complets, soit le droit instantanément d’engager des travaux par exemple. La mise en œuvre de l’acte est immédiate. Les droits sont insérés dès que l’on reçoit le permis.

		La difficulté est qu’au moment de la réception du PC et que l’on affiche, il n’y a pas de délai pour l’afficher. Cet affichage ne donne pas le droit de mettre en œuvre, mais l’affichage va donner une information aux tiers, cela va permettre aux associations, aux riverains d’en avoir connaissance pendant 2 mois continu, va permettre de contester ou de consulter le dossier. Un recours gracieux (auprès du maire) ou contentieux (devant le TA) sera alors possible dans le délai de 2 mois, à compter de l’affichage du permis sur le site.

		Le droit est détenu par l’arrêté pas par l’affichage.

		Permis définitif : 

		-	Exempt de tout recours de tiers (donc gracieux et/ou contentieux)

		-	Exempt de tout retrait administratif : faculté d’une commune par le truchement de son maire de retirer le PC dans un délai de 3 mois à partir de la réception de la notification, s’il estime que le PC comportait des irrégularités qu’il n’a pas eu le temps de voir

		-	Exempt de toute lettre d’observation défavorable du préfet /ou déféré préfectoral. Hypothèse dans laquelle 95% des actes d’urbanisme et d’aménagement doivent être transmis par les CL au préfet pour que ce dernier exerce son contrôle de légalité. La lettre d’observation équivaut à un recours gracieux. Il a un délai de 2 mois à partir de la réception du dossier complet

		-	Lorsque le requérant demande une AJ

		Si les 3 premières hypothèses ne sont pas purgées, le permis n’est pas encore définitif

		Un permis définitif ne donne pas plus de droit. Même un permis non exempt de recours, on peut le transférer, le modifier, le proroger. C’est dans l’expression et mise en œuvre de ce droit que l’on se trouve sécurisé.

		J’obtiens un permis définitif puis un modificatif qui est attaqué. Cela n’interagit pas sur le permis initial. Il y a des droits acquis.

		Les actes réglementaires : 

		Leur opposabilité c’est l’affichage, généralement double affichage par la délibération qui approuve le PLU (élaboré, modifié…) et dans 2 journaux locaux.

		C’est à partir de cette publicité que le changement d’un PLU se fait, sans ces mesures de publicité, le PLU n’est pas opposable.

		Contrairement aux AI, il ne créé pas de droits au regard de la théorie des droits acquis.

		Les promesses de vente, condition suspensive que le PLU ait acquis un caractère définitif. Alors que le PLU n’est jamais définitif.

		Si conteste un PC on peut exciper de l’illégalité du PLU qui a rendu possible la délivrance du PC et donc faire juger que telle ou telle disposition du PLU est illégale.

		Autre manière de contester par la demande d’abrogation : aucun règlement illégal ne doit être maintenu et a fortiori appliqué.

		Or un PLU est un règlement donc possible d’une demande d’abrogation à la collectivité en invoquant des vices de formes ou de fond, par un recours gracieux en conséquence, solliciter abroger le document. Dans la plupart des cas la collectivité n’y défère pas. Si après 2 mois rien, alors refus, alors déféré devant le JA.

		Document réglementaire jamais définitif.

		
	
	\section{La pyramide des normes}
	
		La documentation est thématique. Il y a une abscence de subsidiarité \dots
	% !TeX root = droitUrbanisme.tex

\chapter[Normes nationales et planif. locale]{Normes nationales et planification locale}

\section{Le règlement national d'urbanisme (RNU)}

	L'\articleDu[L]{101-3}{\cu} dispose que le règlement national d'urbanisme (RNU) a vocation à prévoir les règles générales applicables en matière d'utilisation des sols, à l'exception des collectivités d'outre-mer. Ces règles porteront notamment sur la localisation, la desserte, l'implantation et l'architecture des constructions, le mode de clôture et la tenue décente des propriétés foncières et des constructions.

	Ces règles générales, qui constituent le RNU, s'appliquent dans toutes les communes à l'exception des territoires dotés d'une carte communale, d'un plan d'occupation des sols (POS) rendu public ou d'un plan local d'urbanisme (PLU) approuvé ou d'un document en tenant lieu. Toutefois, le RNU peut désigner celles de ces règles qui demeurent applicables sur les territoires couverts par ces documents.

	\medbreak Il constitue à la fois la réglementation d'urbanisme minimale applicable sur tout le territoire en l'absence de règles plus précises et, pour certaines de ses dispositions, une réglementation nationale d'ordre public prévalant sur des règles locales qui ne respecteraient pas ces objectifs.

	Ce double caractère caractérise son champ d'application. Mais, du fait de la généralité de leur champ d'application, les dispositions du RNU ont un contenu généralement assez imprécis et un caractère principalement permissif.
	Elles laissent à l'autorité administrative chargée de les appliquer un large pouvoir d'appréciation. Ce qui donne une place importante au contrôle du juge sur la mise en œuvre du RNU.

	\medbreak La dernière refonte importante du RNU a été réalisée par ordonnance du \printdate{23/09/2015}, suivie de trois décrets en décembre 2015 et janvier 2016\footnote{\cu*, art. R.*\,111-1 s.}. Ce règlement est applicable à toutes les constructions, aménagements, installations et travaux faisant l'objet d'un permis de construire, d'un permis d'aménager ou d'une déclaration préalable ainsi qu'aux autres utilisations du sol régies par le code de l'urbanisme.

	\subsection{Contenu}

	Le RNU dans la rédaction qu'en donne le décret du \printdate{5/1/2007} est divisé en trois sections :
	\begin{itemize}
		\item la section 1 concernant la localisation et la desserte des constructions (\cu*, art. R.\,111-2 à R.\,111-15) ;
		\item la section 2 regroupant les règles relatives à l'implantation et au volume des constructions (\cu*, art. R.\,111-16 à R.\,111-20) ;
		\item et la section 3 étant relative à l'aspect des constructions (\cu*, art. R.\,111-21 à R.\,111-24).
	\end{itemize}

		\subsubsection{Règles relatives à la localisation et à la desserte des constructions}

			Le règlement national d'urbanisme contient tout d'abord des règles relatives à la localisation et à la desserte des constructions, aménagements, installations et travaux. Parmi celles-ci, est d'ordre public celle selon laquelle un projet peut être refusé ou n'être accepté que sous réserve de l'observation de prescriptions spéciales s'il est de nature à porter atteinte à la salubrité ou à la sécurité publique du fait de sa situation, de ses caractéristiques, de son importance ou de son implantation à proximité d'autres installations.

			S'appliqueront sur les territoires non couverts par un PLU les règles selon lesquelles un projet peut être refusé ou accepté sous réserves s'il est susceptible d'être exposé à des nuisances graves, dues notamment au bruit, celle selon laquelle un projet peut être refusé s'il n'est pas correctement desservi par des voies publiques ou privées ou encore la possibilité pour l'autorisation d'occupation du sol d'imposer notamment la réalisation de places de stationnement ou de voies privées nécessaires au respect des règles de desserte des constructions. De même, le RNU contient des règles relatives à la desserte des constructions par les réseaux d'eau potable et d'assainissement ou encore une règle permettant de refuser un projet qui implique la réalisation d'équipements publics hors de proportion avec les ressources de la commune d'implantation.

		\subsubsection{Règles relatives à l'implantation et au volume des constructions}

			Parmi les règles relatives à l'implantation et au volume des constructions, le RNU prévoit qu'une distance d'au moins trois mètres peut être imposée entre deux bâtiments non contigus situés sur un terrain appartenant à un même propriétaire ou encore qu'un bâtiment édifié en bordure de voie publique doit être distant de celle-ci d'au moins la hauteur de ce bâtiment, règles toutes deux applicables sur les territoires non couverts par un PLU. De même, à moins que le bâtiment à construire ne jouxte la limite parcellaire, le RNU prévoit que la distance comptée horizontalement de tout point de ce bâtiment au point de la limite parcellaire qui en est le plus rapproché doit être au moins égale à la moitié de la différence d'altitude entre ces deux points, sans pouvoir être inférieure à trois mètres.

			Le RNU prévoit que des dérogations à ces règles peuvent être accordées par décision motivée de l'autorité compétente en matière d'autorisations d'urbanisme (après avis du maire de la commune dans le cas où celui-ci n'est pas l'autorité compétente).

		\subsubsection{Règles relatives à l'aspect des constructions}

			Le RNU désigne comme d'ordre public la règle selon laquelle le projet peut être refusé ou accepté sous réserves si les constructions, par leur situation, leur architecture, leurs dimensions ou l'aspect extérieur des bâtiments ou ouvrages à édifier ou à modifier sont de nature à porter atteinte au caractère ou à l'intérêt des lieux avoisinants, aux sites, aux paysages naturels ou urbains ainsi qu'à la conservation des perspectives monumentales.

			Dans les secteurs déjà partiellement bâtis, présentant une unité d'aspect, l'autorisation de construire à une hauteur supérieure à la hauteur moyenne des constructions avoisinantes peut être refusée ou subordonnée à des prescriptions particulières.

			Ne s'applique pas, en revanche, sur les territoires couverts par un PLU, la règle selon laquelle les murs séparatifs et les murs aveugles apparentés d'un bâtiment doivent, lorsqu'ils ne sont pas construits avec le même matériau que les murs des façades principales, avoir un aspect qui s'harmonise avec celui des façades.

	\subsection{Effets}

\section[Les GPDU]{Les Grands Principes du Droit de l'Urbanisme (\articleCodifie*{L}{101-1} et \refArticle{L}{101-2})}

	Les principes et objectifs généraux figurent, en droit de l'urbanisme, aux \articlesCodifies{L}{101-1} à \refArticle{L}{ 101-3} du \cu.

	\subsection{Contenu}

		\paragraph{Dispositions de l'article L.\,101-1 du Code de l'urbanisme}

			Les collectivités publiques sont les gestionnaires et les garantes dans le cadre de leurs compétences du territoire français considéré comme le patrimoine commun de la nation.

			L'\articleDu[L]{101-1}{\cu} a pour objectif d'imposer aux différentes collectivités publiques l'harmonisation de leurs prévisions et de leurs décisions d'utilisation de l'espace en renvoyant à la réalisation des objectifs définis à l'\articleDu[L]{101-2}{\cu}.

			Le Conseil d'État a précisé qu'aucune disposition n'impose qu'un plan d'occupation des sols soit compatible avec ceux des communes voisines pour respecter le principe d'harmonisation posé par l'ancien \articleDu[L]{110}{\cu}\footnote{\jurisCE[109114]{24/6/1992}, Cne Pomerols}.

		\paragraph{Différents objectifs de l’article L.\,101-2 du Code de l'urbanisme}

			L'\articleDu[L]{101-2}{\cu} fait référence aux objectifs du développement durable tout en précisant que l'action des collectivités publiques en matière d'urbanisme vise à atteindre sept objectifs énoncés dans cet article. Ces différents objectifs n'ont cessé de s'allonger au fur et à mesure des modifications législatives.
			\begin{quote}
				\itshape
				1° L'équilibre entre :

				a) Les populations résidant dans les zones urbaines et rurales ;

				b) Le renouvellement urbain, le développement urbain maîtrisé, la restructuration des espaces urbanisés, la revitalisation des centres urbains et ruraux ;

				c) Une utilisation économe des espaces naturels, la préservation des espaces affectés aux activités agricoles et forestières et la protection des sites, des milieux et paysages naturels ;

				d) La sauvegarde des ensembles urbains et la protection, la conservation et la restauration du patrimoine culturel ;

				e) Les besoins en matière de mobilité ;

				2° La qualité urbaine, architecturale et paysagère, notamment des entrées de ville ;

				3° La diversité des fonctions urbaines et rurales et la mixité sociale dans l'habitat, en prévoyant des capacités de construction et de réhabilitation suffisantes pour la satisfaction, sans discrimination, des besoins présents et futurs de l'ensemble des modes d'habitat, d'activités économiques, touristiques, sportives, culturelles et d'intérêt général ainsi que d'équipements publics et d'équipement commercial, en tenant compte en particulier des objectifs de répartition géographiquement équilibrée entre emploi, habitat, commerces et services, d'amélioration des performances énergétiques, de développement des communications électroniques, de diminution des obligations de déplacements motorisés et de développement des transports alternatifs à l'usage individuel de l'automobile ;

				4° La sécurité et la salubrité publiques ;

				5° La prévention des risques naturels prévisibles, des risques miniers, des risques technologiques, des pollutions et des nuisances de toute nature ;

				6° La protection des milieux naturels et des paysages, la préservation de la qualité de l'air, de l'eau, du sol et du sous-sol, des ressources naturelles, de la biodiversité, des écosystèmes, des espaces verts ainsi que la création, la préservation et la remise en bon état des continuités écologiques ;

				7° La lutte contre le changement climatique et l'adaptation à ce changement, la réduction des émissions de gaz à effet de serre, l'économie des ressources fossiles, la maîtrise de l'énergie et la production énergétique à partir de sources renouvelables.
			\end{quote}

			Le principe d'équilibre est important. Le Conseil d'État a eu l'occasion d'annuler sur la base de l'erreur manifeste d'appréciation, la création d'une zone NAF (zone de construction future insuffisamment équipée qui peut être urbanisée) à vocation industrielle, artisanale et de services à proximité d'un parc boisé de 40 hectares, classé comme site à protéger, en considérant, notamment, que la révision du POS correspondante ne respectait pas le principe d'équilibre de l'\articleDu[L]{110}{\cu}\footnote{\jurisCE[115248]{21/10/1994}, Cne Bennwihr : JurisData \no 1994-052178 ; Lebon T., p. 1235 ; Dr. adm. 1994, comm. 695}. Le juge a eu l'occasion de vérifier le respect de l'équilibre entre l'extension urbaine, l'exercice des activités économiques et la préservation des milieux, sites et paysages naturels\footnote{\jurisCAA[09NC00452]{Nancy}{11/02/2010}, Cne Berentzwiller et a. : JurisData n° 2010-000719}. Un PLU a été annulé pour violation du principe d'équilibre entre le développement urbain et la gestion économe des espaces naturels et ruraux\footnote{\jurisCAA[10NT02174]{Nantes}{14/05/2012}, Cne Boissy-lès-Perche : JurisData \no 2012-014904}.

	\subsection{Effets}

\section{Les dispositions relatives à la loi montagne et au littoral}

	La loi montagne de 1985 codifiée en majeure partie par les \articlesDuEtSuivants[L]{122-1}{\cu} et littorale L121-1 et suivants
	Loi montagne a des dispositions protectrices du cadre pastorale agricole avec quasi impossibilité de densifier l’urbanisation.
	Ces lois s’imposent au PC, ou sont reprises dans le DTA ou SCOT et DDTU
	Sur la loi montagne il y a un processus les UTN Unité touristique nouvelle : possibilité dans une cogestion entre commune et intercommunalité et préfet, d’organiser une urbanisation importante dans les zones de montagnes.
	Ces UTN est un outil particulier prévu par un SCOT et un PLU et qui doit faire l’objet d’une autorisation UTN cogéré et co instruit par le préfet et la commune pour une urbanisation significative à usage d’habitation de commerce.
	Pour la loi littorale c’est plus complexe, il y a 3 types de zones. Essentiellement dans le CDU, elle s’applique à toutes les commune squi ont au moins une partie de façade au littoral ou un lac de + de 100 hectare
	Champ d’application matériel de la loi littoral : l’ensemble des travaux, quelqu’il soit soit et quel que soit le MO
	Il y a une gradation de contrainte de constructibilité réaménage par la loi ELAN :
	-	L121-16 sauf espace déjà urbanisé, aucune constructibilité dans la bande des 100m, interdiction quasi absolue, sauf deux types de travaux : certains travaux prévus par la loi comme station d’épuration et des voies de dessert, des voies publiques mais très encadrés. Deuxième exception, c’est lorsqu’il y a des activités qui exige la présence de l’eau (une thalasso n’exige pas la présence de l’eau) comme ostréiculture alors il faut justifie une présence obligatoire en limite de l’eau
	-	Deuxième espace entre la bande des 100m et 5-6 km par rapport au plus hautes eaux des rivages on est sur des espaces proches du rivage alors les textes considèrent que seule une urbanisation limitée peut être autorisée L121-13 du CDU qui renvoie à 2 notions la notion d’urbanisation limitée et espace proche du rivage
	Espace proche du rivage : apprécié objectivement par rapport au rivage. S’il y a une falaise, on ne prend pas en considération la falaise
	La covisibilité rivage – terrain, plus il y a une visibilité réciproque forte on peut considérer que l’on est encore dans un espace proche du rivage
	Urbanisation limitée : cela dépend des juridictions certains juges administratifs sont strictes, c’est une appréciation de fait. Le CE a considéré que urbanisation greffé à un port de plaisance arrêt PORTEGA position compréhensive, puis revirement le CE dit non le port de plaisance ne peut plus être l’alibi à a réalisation de constructibilité de complaisance
	Dernière notion, celle de l’extension de l’urbanisation en continuité avec des agglomération et village existant
	On a la bande de 100 m quasi constructibilité, puis espace proche du rivage, donc est ce que l’on peut imaginer une urbanisation plus lourde si extension en continuité par rapport à un village existant L121-8 du CDU seule manière de m’affranchir d’une urbanisation limitée sauf que dans ce cas le JA est exigent sur la notion de village ou agglomération existant : densité suffisamment fort d’urbanisation avec une centralité
	Le JA va regarder au cas par cas la constitution de l’urbanisation existante, densité de maison de voirie, d’équipement public suffisant pour considérer que l’on est en présence d’un village ou agglomération existant.

	L121-13 espace remarquable : espace sur commune littorale (comme parcelle en cravate) la loi littorale s’applique car la commune a une façade littorale, alors ce troisième espace dit remarquable s’appliquera.
	Les espaces remarquables contiennent des espèces animales et végétales propres aux espaces littoraux mais aussi des espaces avec des vestiges ou éléments patrimoniaux propres au littoral. Sur ces espaces, le JA pourra considérer que telle zonage d’un PLU au regard de  cette loi littorale ou tel PC ne sont pas compatibles avec ces espaces remarquables issus de la loi littorales.

	Erreur manifeste en méconnaissant des éléments remarquables qui auraient du être protégés, si EMA par le maire dans le PC alors illégalité alors annulation du PC.


\section{Les schémas directeurs et les nouveaux schémas de cohérences territoriale (SCOT)}

	\subsection{Contenu}

	\subsection{Effets (compatibilité)}

\section{Les Plans locaux d'urbanisme (PLU)}

	\subsection{Contenu des PLU (éléments obligatoires et facultatifs)}

		\subsubsection{Le rapport de présentation} (Articles R151-1 à R151-5)


		\subsubsection{Les orientations d'aménagement et de programmation}
		(Articles R151-6 à R151-8-1)

		\subsubsection{Le règlement}

		... ici reprise du cours ...

		\subsubsection{Les annexes}

		Elles forment la dernière partie du PLU. Elles sont opposables au même titre que les autre éléments du PLU.

		Les PPR sont annexé dans cette partie. Néanmoins, dès lors que ces PPR ne sont pas annexé, ils sont de manière autonome opposable, de la même manière que s'ils s'y trouvait.

	\subsection{Analyse détaillé d'un règlement de PLU}

		Le réglement de PLU est constituée de plusieurs parties décrites aux articles R 151-9 et suivants

		1. un index, ou lexique appelé souvent définition des élements de définition une meilleur compréhension. Ces définitions sont opposables tant au pétitionnaire qu'à l'administration.

		2. les règles générales. Les regles et disposition qui sont suceptible de s'appliquer sur plusieurs zones. Par exemple la reconstruction d'un bâtiment à l'identique. Parfois, on y trouve l'interdiction ... dans un article dédié
		R 151-21 sur la mutualisation des règles de gabarit qui vaut notamment sur les lotissement et les permis valant division. Ces deux catégories de permis particulier si la mutualisation n'a pas été interdite.

		Elle permet d'avoir une bonne combinaison des différentes règles

		3. Les zonages. C'est la partie la plus connue, celle qui permet de manière précise d elocaliser une parcelle et d'identifier rapidement les règles, ou << precriptions >>, qui vont s'appliquer. Ces zones sont réparties à partir des 4 grandes catégories par le curb :
			zone U
			AU
			A
			N
		La définition de ses zones au sein d'un PLU, que ce soit le réglement ou les documents graphiques, sont librement définies sous deux réserves :
			- ne pas faire d'erreur de droit. Par exemple faire apparaitre une incompatibilité avec un document supérieur (un SCOT, ou un zonage qui méconnaitréait des dispositions de la loi littorale
			- ne pas faire d'erreur manifeste d'appréciation (EMA), c'est à dire une erreur grossière relevée par le juge dans la définition du zonage. Par exemple, une zone U d'une commune qui serait limitrophe d'une zone industrielle d'une autre commune.

		\paragraph{Les zone U} R 151-18 (reprise de la définition littérale)

			Les capacité de desserte tant en réseaux qu'en voirie, est suffisante pour ...

			Elles peuvent faire l'objet d'une déclinaison par sous-secteur, et sous-secteur, permettant de mettre en place des règles différenciées.

		\paragraph{Les zone AU} R 151-20 (reprise de la définition littérale) ...

			Zone AU << mixte >> (alinéa 2) ... la constructibilité n'est possible (lotissement, PC valant division)

			Zone AU << stricte >>, (alinéa 3) ... c'est à dire qui ne permet pas en l'état la délivrance de permis. Il sera nécessaire de sortir de cette catégorie.

		\paragraph{Les zone A} R 151-22 (reprise de la définition littérale)

			On peut tout à fait des zones équipées.

			Il y a trois critères qui ne sont pas les seuls :
			- en raison de leur potentiel agronomique
			- ...

		\paragraph{Les zone N} R 151-18 (reprise de la définition littérale)

			équipé ou non

			5 critères ...

			Urbanisation très stricte mais pas impossible. lié à l'activité ou la destination de la zone. sont également admises des extensions de l'urbanisation

			Zone N d'urbanisation diffuse.

		\subsubsection{Les différents articles des zones}

			Jusqu'en 2016, et depuis plus de 40 ans, la nomenclature interne était prévue dans la partie A du code. Un nouve R 151-27 et suivants introduit une nouvelle structuration.

			Dans la pratique, les PLU restent structuré de cette manière.

			Il y avait 15 articles 2 ont été abrogé par la loi ALUR et un est apparu sur les économies d'éarge

			\paragraph{Articles 1 \& 2} Ceux sont les articles qui autorisent, interdisent, organisent ou restreignent les destinations des futurs bâtiments ou à rénover. C’est la vocation générale de la zone.

			Il peut y avoir des destination autorisée mais avec des contraintes.

			Par exemple ...

			R 151-27 et 28 destinations et sous destination

			SINASPIC

			\paragraph{Articles 3 \& 4} relatif à la desserte réseau et voirie. Limite ou non l’accès à la parcelle par sa largeur.

			\paragraph{Article 5} Il a été abrogé par la loi ALUR, à l'exception de 3 communes : Maison Laffite. Cet article encadrait la constructibilité en fonction de la taille des parcelles.

			\paragraph{Articles 6, 7 \& 8} Ce sont les règles de prospect (ou de recul). Ils sont considéré comme simple à expliquer mais difficiles à mettre en œuvre.

			L'article 6 traite des retraites par rapport à une voie public ou un espace public. Toute emprise créant. A priori l'ouverture au public suffit.

			L'article 7 traite du recul par rapport aux autres parcelles privées.

			L'article 8 est relatif aux règles de recul entre deux bâtiments sur une même parcelle. Le plus souvent cet article est plus clément que l'article 7. Les gestes architecturaux pour y échapper ont parfois été sévèrement.

			Les << conventions de cours commune >> permettent d'aménager ces règles en permettant d'appliquer les règles de l'article 8 alors même que ceux sont celles de l'article 7 qui auraient à s'appliquer.

			\paragraph{Article 9} C'est l'article relatif aux emprises qui permet de limiter la superficie du projet, y compris les emprises minérales.

			\paragraph{Article 10} C'est l'article relatif aux hauteurs des bâtiments. Cette hauteur peut être appréhendée au faitage ou à l'égout des toitures.

			Souvent il y a possibilité de dépasser cette hauteur pour un certain nombre d'édicule, si l'auteur du PLU l'a prévu et mentionné.

			...
			La règle que l'on voit communément est de prendre le point médian d'un segment parallèle à la voirie.

			Souvent le point bas se fait à partir du terrain naturel, qui est souvent repéré sur les plans du permis de construire ...

			\paragraph{Article 11} Article relatif aux aspects extérieurs des projets :

			Parfois se limite à une reprise de R 111-27, c'est à dire l'article du RNU relatif ...

			\paragraph{Article 12} relatif aux stationnements (auto, vélo, moto, poussette).  ...

			Les auteurs peuvent rentrer dans un relativement grand détail. Il est possible de : préciser la dimension des places, d'interdire ou autoriser les places commandées, ventiler le ratio des place par destination, \etc Depuis quelques années, la tendance est à la diminution des contraintes au sein de cet article pour diminuer l'importance des voitures en ville.

			possibilité .. par le paiement d'une redevance ... Cette participation n'est plus valable. Il n'est plus possible de s'exonérer de la réalisation des places. Cette disposition a été abrogée car les sommes perçues n'étaient pas utilisées pour réaliser les parking nécessaires.

			Il est possible de s'acquitter e l'obligation de réaliser les places, soit en les réalisant lui-même soit aller les rechercher dans un parking proche (sous forme d'une amodiation) ou, depuis la loi SRU, au sein d'un parking privé proche.

			Si CINASPIC, le plus souvent il n'y a aucun minimum pour les SINASPIC mais simplement la nécessité du montant ... nécessaire

			\paragraph{Article 13}  Article sur les espaces verts et le espace libres. C'est un article qui se décline souvent en plusieurs chapitre.

			Espace Boisé Classé pas forcément boisé ...

			Espace engazonnée. Les auteurs du PLU peuvent aussi exiger que la superficie soit en pleine terre

			Espace libre : par exemple des cours.

			\paragraph{Article 14} Article sur le COS. Il a été abrogé par la loi ALUR.

			\paragraph{Article 14} 3 mots

		\subsubsection{Effet } ...

		Les effets du PLU sont une opposabilité totale dans une relation de conformité, mais également deux regles :
			- regle complementaire, lorsqu'il y a un permis dans un perimetre d'oap, il doit y avoir compatibilité avec les orientations de l'oap
			- le legislateur que sur certain aspects, il y a lieu de pemettre des exceptions R 152-5 à 152-9

	\subsection{Procédure d'élaboration et de révision générale des PLU (les trois phases)}

		L'élaboration d'un PLU les 1ere en 2001 - 2002, suite à la loi SRU de décembre 200 qui a changé des ...

		Les 1er PLU ont été dans leur grande majorité attaqués. Les PLU ont supplanté les POS --- il n'y a plus de POS ou De PAZ applicable.

		Le PLU est l'outils de droit commun ...

		Qu'est ce qui pousse une commune à élaborer ou réviser son PLU. Depuis octobre 2017, il est fait obligation d'appliquer le RNU (R 111-1 et suivants) aux communes ayant toujours un POS. Les communes sont donc vivement incités à élaborer un PLU. Il y a certaine région, la Corse par exemple,  qui ont fait le choix de rester en RNU.

		La

		\paragraph{Réflexion sur le périmètre} Contrairement ax anciens POS, il n'est pas possible d'élborer un PLU partiel. Le PLU doit nécessairement couvrir l'intégralité du territoire communal --- ou intercommunal L153-1.

		Deux exceptions :
			- L 153-3, lorsqu'il y a fusion d'EPCI
			- plus fréquemment, lorsqu'il y a annulation partielle, ou déclaration d'illégalité partielle, d'un PLU.
			(12 mn environ) %L'annulation partielle peut provenir d'une décision d'un juge site à un recours sur un PC.

			Dans ce cas deux conséquences : le zonage ou l'article annulé ... on reprend le doc d'urbanisme antérieur.
			(zut j'ai décroché)

			La déclaration (exception )d'illégalité pas de possibilité de le conserver .. = paralysie

			Dans ces deux cas nous avons une disparition.

			Ce qui peut poser problème si des pétitionnaires << s'engouffrent dans la brèche >> et déposent des demande de PC. Il faut que la commune fasse vite.

			3 phaseq :
			(début partie 2)
			prescription
			arrert

		\paragraph{Première délibération}

		Article L 153-11 << cite
		Le Maire est incompétent pour lancer c'est le Conseil Municipal. On trouve dans la délib :
		 - objectifs poursuivis (protection, développement économique, intensification des liaisons douces, \etc)
		 - les modalités de la concertation. Elles sont librement définies par la collectivité. Fréquemment on trouve :
		 	- des réunions publiques
		 	-
		 	- un premier calendrier prévisionnel
		Il n'y a pas  de durée particulière. Ce qui est exigé est que la concertation s'arrête, et que le bilan soit tiré, avant l'enquête publique.

		Le 3\ieme{} alinéa de 143-11 prévoit la possibilité de << sursoir à statuer >> dès lors que des demandes d'autorisation seraient contraire aux prescriptions du futur PLU. En pratique :
			- Cette possibilité n'est possible que dans l'élaboration d'un PLU, ou une révision générale ;
			- Cette possibilité n'est possible qu'après la délibération qui consacre le débat sur les orientations générales du projet d'aménagement et de développement durable. modification en 2017 tenant compte de la jurisprudence pour ajouter --- auparavant il était possible de sursoir à statuer dès lors que la délibération ... l'élaboration du PLU ;

		\paragraph{Deuxième délibération sur le PADD.}

		Cela suppose qu'il y ait eu un débat, dont on conserve une trace.

		A partir de ce délibération, le sursis à statuer devient possible. Le juge reste souverain pour estimer si l à pris une consistance suffisante. La difficulté pour le pétitionnaire est qu'en dehors des moments où émerge ... et jusqu'à l'enquête publique, peu d'information filtre sur l'avancée des travaux d'élaboration. Le pétitionnaire est donc dans l'incertitude sur le niveau d'avancement. Plus le temps passe, plus il est probable que le sursis à statuer soit fondé.

		\paragraph{Arrêté de projet du PLU } Lorsque le document est

		Communication aux PPA L 153-16

		L 153-17 plus pernicieux car il évoque l'obligation de communiquer, à leur demande, à d'autres collectivités locales. Pour demander, il faut connaitre. Donc il faut informer pour pouvoir prouver qu'ils étaient au courant.

		Délais de 2 mois.

		Réception de l'ensemble des avis des PPA.

		Si il. La collectivité peut décider d'intégrer les observations.

		Si ces ajustements sont purmeent conformes. S'il y a des modifications substantielles -> nouvelles consultations des PPA. Ces avis ne sont pas publics à ce stade.

		\paragraph{Préparation de l'enquête publique}

		elle est important car elle va consacrer le PLU dans un état quasi définitif. Il ne pourra y avoir que des modifications mineures, pas de reise en cause de l'économie générale.

		Le calendrier ordinaire :

		Désignation du commissaire enquêteur. Le président du TA compétent est contacté par le maire pour désignation d'un commissaire enquêteur.
		A été jugé comme non impartial le mire adjoint d'une commune limitrophe.
		Le maire reçoit ourrier, puis

		Arrête avec le commissaire enquêteur les dates (au minimum 1 mois) et les modalités pratique ainsi que les dates de permanence du commissaire enquêteur. Parfois nomination d'une commission d'enquête.

		Ensuite information en deux temps. Plusieurs semaines dans la presse locale, en mairie, en mairie annexe, mentionnant l'ensemble des

		Code de l'environnement. Il y a eu énormément de contentieux avec annulation de PLU sur ce sujet.

		Modalité de l'enquête : accès libre et le plus large possible.
		Obligation d'un registre d'enquête public, qui doit obéir à des règles formelles très stricte.

		On trouve le PLU complet, l'ensemble des avis et les courriers d'envoie avec les dates de transmission, le porter à connaissance ...

		(1h22)

		.. les administrés de la commune, les administrés des communes voisines, les associations, les personnes morales de la communes, les autres collectivités locales \etc il n'y pas de limite. Il est possible de déposer des observations anonymes.

		Depuis quelques années, il y a en parallèle une consultation sur Internet.

		Fermeture des débats, par le commissionnaire enquêteur. Qui collationne . sauf s'il considère que l'enquête publique doit être prorogé. En cas de désaffection importante de la population, ou à l'inverse si l'enquête.

		(1:28)

		Le commissaire enquêteur est souverain de proroger ou non.

		Prépare 2 documents : le rapport et les conclusions.

		Dans le rapport, relate de manière fidèle toute l'enquête ... \etc Ne touche pas au fond du sujet.

		Pour la rédaction du rapport, en pratique est de rentrer dans le contenu des observations et de proposer à la collectivité une réunion de travail. Toute les questions

		(1:36) << une réponse personnalisée et circonstanciée >>

		%Si annulation du PLU alors responsabilité de la commune.

		Le calendrier de restitution : délai d'un mois, mais pas de sanction.

		Le commissaire peu rendre des conclusions favorables, sans réserves ni recommandations. C'est de plus en plus rare.
		Il peut émettre un avis favorable avec des recommandations.
		Il peut émettre un avis favorable avec des réserves ou des réserves exprès. L'avis ne vaut que si les réserves sont prises en considération. Sinon il faut considérer l'avis comme défavorable.
		Il peut émettre un avis défavorable.

		Juridiquement la commune peut passer outre les réserves. Mais cela renforce la possibilité de recours. Cela permet au Préfet de demander une suspension en urgence sans avoir à démontrer l'urgence.

		2 critères :
			- non substantiel
			- trouve leur source

		\paragraph{Délibération d'approbation}

		On considère que s'il y a un SCOT, il y a aucune raison pour que le PLU ne soit pas exécutoire


	\subsection{Les autres procédures d'adaptation des PLU}

		Il ne faut pas utiliser << modifcation >>
		 Du plus lourd au plus léger.

		 \subsubsection{Procédure de révision} L 153-31

		 \begin{quote}
		 	<< { \itshape
		 		Le plan local d'urbanisme est révisé lorsque l'établissement public de coopération intercommunale ou la commune décide :

		 		1\degres Soit de changer les orientations définies par le projet d'aménagement et de développement durables ;

		 		2\degres Soit de réduire un espace boisé classé, une zone agricole ou une zone naturelle et forestière ;

		 		3\degres Soit de réduire une protection édictée en raison des risques de nuisance, de la qualité des sites, des paysages ou des milieux naturels, ou d'une évolution de nature à induire de graves risques de nuisance.

		 		4\degres Soit d'ouvrir à l'urbanisation une zone à urbaniser qui, dans les neuf ans suivant sa création, n'a pas été ouverte à l'urbanisation ou n'a pas fait l'objet d'acquisitions foncières significatives de la part de la commune ou de l'établissement public de coopération intercommunale compétent, directement ou par l'intermédiaire d'un opérateur foncier.

		 		5\degres Soit de créer des orientations d'aménagement et de programmation de secteur d'aménagement valant création d'une zone d'aménagement concerté.
		 	} >>
		 \end{quote}

	 	Les 3 1\ieres{} hypothèses sont les plus importantes.

	 	Que dit le juge du 1. S'il y a changement d'une ou plusieurs orientation du PADD alors il s'agit d'une révision.

	 	Le 2 existait déjà du temps des POS. Dès lors que vous

	 	Le 3 est presque subjectif. C'est un PPRI

	 	\subsubsection{Procédure de modification} L 153-36

		 	\paragraph{La modification de droit commun} L 153-41

		 		\begin{quote}
		 			<< {\itshape
		 				Le projet de modification est soumis à enquête publique réalisée conformément au chapitre III du titre II du livre Ier du code de l'environnement par le président de l'établissement public de coopération intercommunale ou le maire lorsqu'il a pour effet :

		 				1\degres Soit de majorer de plus de 20 \% les possibilités de construction résultant, dans une zone, de l'application de l'ensemble des règles du plan ;

		 				2\degres Soit de diminuer ces possibilités de construire ;

		 				3\degres Soit de réduire la surface d'une zone urbaine ou à urbaniser ;

		 				4\degres Soit d'appliquer l'article L. 131-9 du présent code.
		 			} >>
		 		\end{quote}

	 			Les 3 1\iers{} cas sont les plus importants.
	 			\begin{enumerate}
	 				\item Le mot zone n'est pas limité à une mais peut concerner un secteur ou un sous-secteur. Il faut apprécier si la modification des règles gabaritaires\footnote{articles 6, 7, 8, 9, 10 et 13 du règlement}

	 				\item  valable pour toute les zones, même en zone N

	 				\item Si on réduit la surface .
	 			\end{enumerate}

	 		Souvent 6 ou 7 mois

	 		\paragraph{La modification simplifiée} L 153-45

		 		\begin{quote}
		 			<< {\itshape
		 				La modification peut être effectuée selon une procédure simplifiée :

		 				1\degres Dans les cas autres que ceux mentionnés à l'article L. 153-41 ;

		 				2\degres Dans les cas de majoration des droits à construire prévus à l'article L. 151-28 ;

		 				3\degres Dans le cas où elle a uniquement pour objet la rectification d'une erreur matérielle.

		 				Cette procédure peut être à l'initiative soit du président de l'établissement public de coopération intercommunale ou du maire d'une commune membre de cet établissement public si la modification ne concerne que le territoire de cette commune, soit du maire dans les autres cas.
		 			} >>
		 		\end{quote}

			 	L'erreur matérielle. La plus grave et la plus courante est l'erreur du fonds de plan.

			 	Le cas principal est le 2\ieme{}. Les effets sont importants, tant opérationnellement qu'en terme de contentieux. L  L 153-47 prévoit que les documents << sont mis à disposition du public pendant un mois, dans des conditions lui permettant de formuler ses observations >> sans le formalisme de l'enquête publique.

			 	Cependant prête le flan à la critique du détournement de pouvoir si la modification est trop ciblée, ou, à l'inverse, si la modification est trop large à vous rapprocher du cas de la modification de droit commun.

			 	On observe souvent un (24:30)

			 	Souvent 3 ou 4 mois pour faire une modification simplifiée.

			 	Très intéressant pour tout ce qui ne rentre pas dans le champ de la modification classique. Ex. : réduire les ratios de stationnement.

		 	\subsubsection{Procédure de mise en compatibilité} L 153-49

		 		\paragraph{Déclaration de projet}

		 		Le véritable intérêt résulte de la pratique. C'est l'adaptation à un projet.

		 		Le temps d'élaboration est de 6 à 7 mois

		 		Très souvent consiste en un secteur de plan masse\footnote{
		 			article R 151-40 « Dans les zones U, AU,
		 			dans les secteurs de taille et de capacité d'accueil limitées délimités en application de
		 			l'article L. 151-13, ainsi que dans les zones où un transfert des possibilités de construction a
		 			été décidé en application de l'article L. 151-25, le règlement peut définir des secteurs de plan
		 			masse côté en trois dimensions »

			 		Le plan de masse fixe les volumes constructibles en déterminant a minima les règles
			 		d’implantation, d’emprise et de hauteur. Il doit faire apparaître avec suffisamment de
			 		précision l’implantation des constructions projetées
			 		, sans avoir pour autant à préciser leur
			 		localisation exacte.
		 		}.

	\subsection{Les certificats d'Urbanise (CU)}

\section{Les cartes communales}

\section{L'impact du droit de l'environnement}

	\subsection{L'enquête publique (PLU et certains PC)}

	\subsection{Étude d'impact et évaluation environnementale}

	\subsection{Les mesures favorables aux énergies renouvelables}

	\chapter{L'aménagement}
	\chapter{Fiscalité de l'urbanisme}

Deux propos :
\begin{itemize}
	\item On parle tout d'abord de paticpation d'urba, \cad de qui assujettissent
	
	le droit commun est articulé en trois 
	l\begin{enumerate}
		\item 'outil par défaut la taxe d'aménagment qui est venu remplacer la Taxe Local d'Équipement ;
		\item qui peut être écarté, pour ce qui est de la part communale de la T,  par la ZAC et le régime de participation ;
		\item ou être écarté dans les mêmes dans le cadre d'un \PUP
	\end{enumerate}

	\item il n'y pas d'autres , et donc de possibilité pour une \commune de demander une taxe qui ne trouverait pas dans l'\articleCU{L}{332-6}
	\begin{quote}
		<< {Les bénéficiaires d'autorisations de construire ne peuvent être tenus que des obligations suivantes :
			
			1° Le versement de la taxe d'aménagement prévue par l'article L. 331-1 ou de la participation instituée dans les secteurs d'aménagement définis à l'article L. 332-9 dans sa rédaction antérieure à l'entrée en vigueur de la loi n° 2010-1658 du 29 décembre 2010 de finances rectificative pour 2010 ou dans les périmètres fixés par les conventions visées à l'article L. 332-11-3 ;
			
			2° Le versement des contributions aux dépenses d'équipements publics mentionnées au c du 2° de l'article L. 332-6-1, \lips 
			
			3° La réalisation des équipements propres mentionnées à l'article L. 332-15 ;
			
			4° Le versement pour sous-densité prévu aux articles L. 331-36 et L. 331-38 ;
			
			5° Le versement de la redevance d'archéologie préventive prévue aux articles L. 524-2 à L. 524-13 du code du patrimoine.} >>
	\end{quote}
\end{itemize}

\section{Taxe d'aménagement}

	\begin{quote}
		\articleCU{L}{331-6}
		
		<< {\itshape Les opérations d'aménagement et les opérations de construction, de reconstruction et d'agrandissement des bâtiments, installations ou aménagements de toute nature soumises à un régime d'autorisation en vertu du présent code donnent lieu au paiement d'une taxe d'aménagement, sous réserve des dispositions des articles L. 331-7 à L. 331-9.
			
		Les redevables de la taxe sont les personnes bénéficiaires des autorisations mentionnées au premier alinéa du présent article ou, en cas de construction sans autorisation ou en infraction aux obligations résultant de l'autorisation de construire ou d'aménager, les personnes responsables de la construction.
		
		Le fait générateur de la taxe est, selon les cas, la date de délivrance de l'autorisation de construire ou d'aménager, celle de délivrance du permis modificatif, celle de la naissance d'une autorisation tacite de construire ou d'aménager, celle de la décision de non-opposition à une déclaration préalable ou, en cas de constructions ou d'aménagements sans autorisation ou en infraction aux obligations résultant de l'autorisation de construire ou d'aménager, celle du procès-verbal constatant l'achèvement des constructions ou des aménagements en cause.} >>
	\end{quote}

	\paragraph{Champ d'application} Elle s'applique pour les ...
	
	\paragraph{Redevable} : personne 
	
	\paragraph{Fait générateur} ... 
	
	L'idée est que la taxe est due si ... En cas de retrait du permis pour cause de fraude, les sommes déjà versée par le titulaire lui sont dues.
	Le retrait à tout moment à la demande du titulaire ... remboursement 
	Permis annulée : remboursement
	PCM ...
	
	Transfert de permis. Encore une fois plusieurs cas possibles...
	
	\paragraph{Base imposable }
	
	\begin{quote}
		\textbf{\articleCU{L}{331-10}}
		
		<< {\itshape L'assiette de la taxe d'aménagement est constituée par :
			
			1° La valeur, déterminée forfaitairement par mètre carré, de la surface de la construction ;
			
			2° La valeur des aménagements et installations, déterminée forfaitairement dans les conditions prévues à l'article L. 331-13.
			
			La surface de la construction mentionnée au 1° s'entend de la somme des surfaces de plancher closes et couvertes, sous une hauteur de plafond supérieure à 1,80 mètre, calculée à partir du nu intérieur des façades du bâtiment, déduction faite des vides et des trémies.} >>
	\end{quote}
	On voit que  la bas de la TA n'est pas la SDP
	
	\subsection{Objectifs de participation de droit commun}
	
	\subsection{Les trois parts de la TA et détermination de l'assiette}
	
	\subsection{Exonération et mise en recouvrement}
	
\section{Convention de Projet Urbain Partenarial}

	Descendant du Programme d’Aménagement d’Ensemble (PAE)
	
	Le projet urbain partenarial (PUP) permet aux communes, aux établissements publics, au représentant de l’État dans le cadre d’opérations d’intérêt national (OIN) ou à certaines collectivités territoriales ou établissements publics (article L. 312-3) dans le périmètre d'une grande opération d'urbanisme, d’assurer le préfinancement d’équipements publics par des personnes privées (propriétaires fonciers, aménageurs ou constructeurs) via la conclusion d’une convention.
	
	L’initiative de cette convention appartient à la collectivité compétente en matière d’urbanisme ou aux porteurs de projet (aux propriétaires fonciers, constructeurs ou aménageurs) qui la proposent si leur projet nécessite la réalisation d’équipements publics difficiles à financer par la seule taxe d’aménagement (TA).
	
	Depuis le marquage au sol quelques milliers d'euros, jusq'à des presques ZAC
	
	\begin{quote}
		\textbf{\articleCU{L}{332-11-3}}
		
		<< {\itshape I.-\textbf{Dans les zones urbaines et les zones à urbaniser délimitées par les plans locaux d'urbanisme ou les documents d'urbanisme en tenant lieu}, lorsqu'une ou plusieurs opérations d'aménagement ou de construction nécessitent la réalisation d'équipements autres que les équipements propres mentionnés à l'article L. 332-15, une convention de projet urbain partenarial prévoyant la prise en charge financière de tout ou partie de ces équipements peut être conclue entre les propriétaires des terrains, les aménageurs, les constructeurs et :
			
		1° Dans le périmètre d'une opération d'intérêt national au sens de l'article L. 102-12, le représentant de l'Etat ;
		
		2° Dans le périmètre d'une grande opération d'urbanisme au sens de l'article L. 312-3, la collectivité territoriale ou l'établissement public cocontractant mentionné au même article L. 312-3 ;
		
		3° Dans les autres cas, la commune ou l'établissement public compétent en matière de plan local d'urbanisme.
		
		II.-Lorsque des équipements publics ayant vocation à faire l'objet d'une première convention de projet urbain partenarial desservent des terrains autres que ceux mentionnés dans le projet de ladite convention, par décision de leur organe délibérant, la commune ou l'établissement public compétent en matière de plan local d'urbanisme, ou la collectivité territoriale ou l'établissement public cocontractant mentionné à l'article L. 312-3 dans le périmètre des grandes opérations d'urbanisme ou le représentant de l'Etat par arrêté, dans le cadre des opérations d'intérêt national, fixe les modalités de partage des coûts des équipements et délimite un périmètre à l'intérieur duquel les propriétaires fonciers, les aménageurs ou les constructeurs qui s'y livrent à des opérations d'aménagement ou de construction participent, dans le cadre de conventions, à la prise en charge de ces mêmes équipements publics, qu'ils soient encore à réaliser ou déjà réalisés, dès lors qu'ils répondent aux besoins des futurs habitants ou usagers de leurs opérations. Les conventions successivement établies peuvent viser des programmes d'équipements publics différents lorsque les opérations de construction attendues dans chaque périmètre de convention ne nécessitent pas les mêmes besoins en équipements.
		
		Le périmètre est délimité par délibération du conseil municipal ou de l'organe délibérant de l'établissement public ou, dans le cadre des opérations d'intérêt national, par arrêté préfectoral, pour une durée maximale de quinze ans. Le périmètre est délimité par décision de l'organe délibérant de la collectivité territoriale ou de l'établissement public cocontractant mentionné au même article L. 312-3 dans le périmètre des grandes opérations d'urbanisme, pour une durée pouvant être supérieure à quinze ans sans pour autant pouvoir excéder la durée fixée par l'acte décidant de la qualification de grande opération d'urbanisme.
		
		III.-Avant la conclusion de la convention, les personnes ayant qualité pour déposer une demande de permis de construire ou d'aménager peuvent demander à la commune ou à l'établissement public de coopération intercommunale compétent en matière de plan local d'urbanisme ou à la collectivité territoriale ou l'établissement public mentionné audit article L. 312-3 dans le périmètre des grandes opérations d'urbanisme ou au représentant de l'Etat dans le cadre des opérations d'intérêt national qu'ils étudient le projet d'aménagement ou de construction et que ce projet fasse l'objet d'un débat au sein de l'organe délibérant. L'autorité compétente peut faire droit à cette demande.
		
		La demande est assortie d'un dossier comportant la délimitation du périmètre du projet d'aménagement ou de construction, la définition du projet ainsi que la liste des équipements publics à réaliser pour répondre aux besoins des futurs habitants ou usagers des constructions à édifier dans le périmètre.
		
		Cette convention ne peut mettre à la charge des propriétaires fonciers, des aménageurs ou des constructeurs que le coût des équipements publics à réaliser pour répondre aux besoins des futurs habitants ou usagers des constructions à édifier dans le périmètre fixé par la convention ou, lorsque la capacité des équipements programmés excède ces besoins, la fraction du coût prod'portionnelle à ceux-ci.
		
		La convention fixe les délais de paiement. La participation peut être acquittée sous forme de contribution financière ou d'apports de terrains bâtis ou non bâtis.
		
		La convention peut prévoir que la contribution financière prévue à l'avant-dernier alinéa du présent III est versée directement à la personne publique assurant la maîtrise d'ouvrage des équipements publics mentionnés au troisième alinéa du présent III.} >>
	\end{quote}
	
	carte communale n'est pas un document tenant lieu de PLU. Donc toute le monde sauf RNU et CC.
	
	La convention de PUP ne peut être signée que dans les communes dotées d’un plan local d’urbanisme (PLU) ou d’un document en tenant lieu tel qu’un plan d’occupation des sols (POS) ou un plan de sauvegarde et de mise en valeur d'un secteur sauvegardé (PSMV). Dans ces communes, la convention PUP ne peut être signée que dans les zones urbaines ou les zones à urbaniser \cad les zones U et AU des PLU et U et NA des POS.
	
	Il faut comprendre comme << lotissement >>
	
	Deux types de \PUP : \begin{itemize}
		\item avec une délib., hypothèse de définition de périmètre et validation de la convention
		\item avec deux délib., hypothèse en deux temps
	\end{itemize}

	\subsection{Objectifs et partis à la convention}
	
		La convention fixe toutes les modalités de participation au financement des équipements publics, notamment les montants, et les délais de paiement.
		
		Doivent y figurer :
		\begin{itemize}
			\item la liste des équipements à financer, leur coût provisionnel et les délais de réalisation ;
			\item le montant de la participation à la charge du constructeur ou aménageur ;
			\item le périmètre de la convention (qui correspond aux terrains d’assiette de l’opération d’aménagement et de construction et non aux seuls équipements publics à réaliser) ;
			\item les modalités et délais de paiement. La participation peut prendre la forme d’une contribution financière ou d’un apport de terrain bâti ou non bâti ;
			\item la durée d’exonération de la part communale de la taxe d’aménagement, qui ne pourra pas excéder dix ans.
		\end{itemize}
		
		Le quantum est librement défini, sous réserve de proportionnalité, et l'échéancier de versement sera librement convenu entre 
		
		\paragraph{Partis à la convention} Les parties à la convention de PUP peuvent être
		exclusivement :
		\begin{itemize}
			\item le représentant de la \commune compétente en matière de PLU,
				\footnote{Ou le représentant de l’État pour les conventions à établir dans le cadre d’une opération
				d’intérêt national. Dans ce  cas de figure, la convention peut être tripartite si la commune doit réaliser certains équipements} ;
				
			\item les porteurs de projets, privés ou publics
				(aménageurs, lotisseurs, propriétaires fonciers
				ou constructeurs) qui projettent de déposer, à
				terme, une demande d’autorisation d’occuper
				le sol sur une assiette foncière qu’ils maîtrisent.
			
			Ces personnes peuvent se présenter en indivision. Les textes ne font pas de distinction entre
			la nature publique ou privée des personnes
			concernées. Il suffit que leurs projets soient
			consommateurs du programme des équipements à financer. 
		\end{itemize}
	
	\subsection{Équipement public de la PUP}
	
		Ils sont totalement libres : infrastructure ou superstructure
		
		Les équipements publics financés par les constructeurs sont ceux qui, non seulement sont rendus nécessaires par les opérations de construction ou d’aménagement initiées par ces derniers, mais répondent aussi aux besoins des futurs habitants ou usagers du projet.
	
	\subsection{Modalité de contribution}
	
\section{Participations de ZAC et convention de participation}
	
	La ZAC se conduit selon deux phases : une phase de création et une phase de réalisation (cf ).
	
	La participation pour le financement de ses équipements doit respecter un principe général : la participation de l’aménageur doit être proportionnelle et nécessaire aux besoins de l’opération.
	
	La répartition des coûts et la responsabilité en matière de maîtrise d’ouvrage est fixée dans la convention signée entre les acteurs impliqués dans la réalisation de la ZAC. La participation est inscrite dans le bilan financier de l’opération.
	
	En ZAC, la participation est perçue par l’aménageur au titre de la charge foncière puisqu’il a vocation à céder, louer ou concéder l’assiette foncière des terrains qu’ils maîtrisent.
	
	Pour les terrains non maîtrisés par l’aménageur ou qu’il n’a pas vendu ou cédé une convention de participation doit être conclue entre le constructeur et le lotisseur et la commune pour préciser les conditions dans laquelle ces derniers participent aux coûts d’équipement de la zone.
	
	Cette convention constitue une pièce obligatoire du permis de construire ou d’aménager.
	
	L’aménageur peut également, dans le cadre de sa négociation avec les propriétaires de terrains situés dans la ZAC, établir avec eux une convention foncière définissant les conditions dans lesquelles ils participent à l’aménagement (emprise foncière, cession de terrains, établissement de servitudes, ouverture à l’urbanisation, etc). Cette convention est distincte de la convention de participation financière prévue ci-dessus.

\section{Taxes particulières}
	\chapter{Permis de construire}

\section{Caractères généraux}

\section{Contenu du dossier de demande et pièces complémentaires}

\section{Déroulement et incidents dans la phase d'instruction}

\section{Résultats de l'instruction}

\section{Les différents type de permis de construire}

	\subsection{PC et PCM}
	
	\subsection{PC rectificatif}
	
	\subsection{PC de régularisation}
	
	\subsection{PC valant division (article R.~431-24)}
	
	\subsection{PC précaire}
	
\section{Transfert de PC}

	\subsection{Transfert de droit commun et total}
	
	\subsection{Transfert partiel}
	
\section{Prorogation du PC}
	
	\subsection{Prorogation sans démarrage du chantier}
	
	\subsection{Prorogation du fait des travaux engagés}
	
\section{Retrait du PC}
	
	\subsection{Principe et modalités du retrait}
	
	\subsection{Retrait à tout moment dans l'hypothèse d'un PC obtenu grâce à des manœuvres commises par le pétitionnaire}
	
\section{Le régime de la conformité administrative (totale ou partielle)}
	\chapter[Lotissements]{Lotissements : déclaration préalable et permis d'aménager}


\section{Les différents modes de division foncières (hors lotissement)}


\section{Contenu et effets des déclarations préalables et permis d'aménager}
	
	
	\chapter[Contentieux]{Les contentieux de l'urbanisme et de l'aménagement}

\section{Situations litigieuses : contentieux ou protocole transactionnel ?}

\section{Les contentieux administratifs}

	\subsection{Le contentieux de la légalité}
		
		\subsubsection{Gestion de l'affichage et des délais}
		
		\subsubsection{Les moyens de recevabilité des actions des requérants}
		
		\subsubsection{Moyens juridiques}
		
		\subsubsection{Pratique du référé-suspension}
		
		\subsubsection{Pratique de l'annulation partielle et de la régularisation}
		
	\subsection{Le contentieux indemnitaire}
	
		\subsubsection{Les 3 fondements de la responsabilité}
		
		\subsubsection{Les différents préjudices indemnisables}
		
		\subsubsection{Les causes exonératoires}
		
		\subsubsection{Le processus de mise en œuvre de la responsabilité de l'Administration}
		
\section{Le contentieux pénal}

\section{Le contentieux civil}

	\tableofcontents

\end{document}
