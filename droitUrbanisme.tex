\documentclass[10pt,a4paper,twoside]{book}

\usepackage{makeidx}
\usepackage[utf8]{inputenc}
\usepackage[T1]{fontenc}
\usepackage{graphicx}
\usepackage{eurosym}
\usepackage{enumitem}
\usepackage{xspace}
\usepackage{moredefs}\usepackage[mla]{lips}
% à passer en dernier
\usepackage[french]{babel,varioref}
\usepackage{hyperref}

\author{Me Bernard \nom{Larmorlette}}
\title{Droit de l'Urbanisme et de l'Aménagement}
\date{Février 2020}

\hypersetup{%
%	pdfinfo={%
%		Title={Droit de l'Urbanisme et de l'Aménagement}%
%		, Subject={}%Marché \@ReferenceMarche}%
%		, Author={Samuel Déom}%
%		%, Keywords={}%\@ReferenceMarche}%
%	}%
	, colorlinks = true% colore, plutot qu'encadre, les liens hypertexte
	, linkcolor = black% colore les liens internes en noir
	, urlcolor = black% colore les liens externes en noir
	, breaklinks = true% autorise les liens à être étendus sur plusieurs lignes
}

% Ecrire du texte juridique
\makeatletter
\newcommand*{\assPlen}{\@ifstar{\mbox{ass. plén.}\xspace}{assemblée plénière\xspace}}
\newcommand*{\civUn}{\@ifstar{civ. 1\iere{}\xspace}{première chambre civile\xspace}}
\newcommand*{\civDeux}{\@ifstar{civ. 2\ieme{}\xspace}{deuxième chambre civile\xspace}}
\newcommand*{\civTrois}{\@ifstar{civ. 3\ieme{}\xspace}{troisième chambre civile\xspace}}
\newcommand*{\CourDeCas}{\@ifstar{Cass.\xspace}{Cour de Cassation\xspace}}
\newcommand*{\CA}{\@ifstar{CA\xspace}{Cour d'Appel\xspace}}
\newcommand*{\CE}{\@ifstar{CE\xspace}{Conseil d'État\xspace}}
\makeatother

\newcommand*{\jurisCourDeCas}[3][]{\CourDeCas* #2, \printdate{#3}\ifthenelse{\equal{#1}{}}{}{, \no#1}}% Numéro de chambre, date et en option no de pourvoi
\newcommand*{\jurisCA}[2]{\CA* #1, \printdate{#2}}%Ville, date

\newcommand*{\refArticle}[2]{\mbox{#1.\,#2}}
\newcommand*{\articleCodifie}[2]{\mbox{article~\refArticle{#1}{#2}}}
\newcommand*{\articlesCodifies}[2]{\mbox{articles~\refArticle{#1}{#2}}}
\newcommand*{\ArticleCodifie}[2]{\mbox{Article~\refArticle{#1}{#2}}}
\newcommand*{\ArticlesCodifies}[2]{\mbox{Articles~\refArticle{#1}{#2}}}
\newcommand*{\articlesCodifiesEtSuivants}[2]{\articlesCodifies{#1}{#2} et suivants}
\newcommand*{\articleDu}[3][]{\ifthenelse{\equal{#1}{}}{\mbox{article~#2}}{\articleCodifie{#1}{#2}} du #3}
\newcommand*{\articlesDu}[3][]{\ifthenelse{\equal{#1}{}}{\mbox{articles~#2}}{\articlesCodifies{#1}{#2}} du #3}
\newcommand*{\articlesDuEtSuivants}[3][]{\ifthenelse{\equal{#1}{}}{\mbox{articles~#2}}{\articlesCodifies{#1}{#2}} et suivants du #3}
\newcommand*{\ArticleDu}[3][]{\ifthenelse{\equal{#1}{}}{\mbox{Article~#2}}{\ArticleCodifie{#1}{#2}} du #3}
\newcommand*{\ArticlesDu}[3][]{\ifthenelse{\equal{#1}{}}{\mbox{Articles~#2}}{\ArticlesCodifies{#1}{#2}} du #3}
\newcommand*{\ArticlesDuEtSuivants}[3][]{\ifthenelse{\equal{#1}{}}{\mbox{Articles~#2}}{\ArticlesCodifies{#1}{#2}} et suivants du #3}

\makeatletter
\newcommand*{\ca}{\@ifstar{\mbox{C.~assur.}\xspace}{Code des assurances\xspace}}
\newcommand*{\cch}{\@ifstar{CCH\xspace}{Code de la construction et de l'habitation\xspace}}
\newcommand*{\cciv}{\@ifstar{\mbox{C.~civ.}\xspace}{Code civil\xspace}}
\newcommand*{\ccom}{\@ifstar{\mbox{	C.~com.}\xspace}{Code du commerce\xspace}}
\newcommand*{\ccons}{\@ifstar{\mbox{C.~consom.}\xspace}{Code de la consommation\xspace}}
\newcommand*{\ccp}{\@ifstar{CCP\xspace}{Code de la commande publique\xspace}}
\newcommand*{\cpc}{\@ifstar{CPC\xspace}{Code de procédure civile\xspace}}
\newcommand*{\cpi}{\@ifstar{CPI\xspace}{Code de la propriété intellectuelle\xspace}}
\newcommand*{\cu}{\@ifstar{\mbox{C.~urb.}\xspace}{Code de l'urbanisme\xspace}}
\makeatother
\newcommand*{\articleCodifie}[2]{article~#1.\,#2}
\newcommand*{\articlesCodifies}[2]{articles~#1.\,#2}

% Conservé pour compatibilité avec première version
\newcommand*{\articleCU}[2]{\articleDu[#1]{#2}{\cu}\index{#1#2@#1.\,#2}}
\newcommand*{\articlesCodifiesS}[2]{articles~#1.\,#2 et suivants}

% Commandes utiles souvent
\newcommand*{\cad}{c'est-à-dire\xspace}
\newcommand*{\commune}{collectivité locale\xspace}
\newcommand*{\CM}{Conseil Municipal\xspace}
\newcommand*{\CU}{Code de l'urbanisme\xspace}
\newcommand*{\etc}{\emph{etc}.\xspace}
\newcommand*{\metreCarre}{m\up{2}\xspace}
\newcommand*{\montant}[1]{\nombre{#1}~\euro}
\newcommand*{\nom}[1]{\textsc{#1}}
\newcommand*{\pourcent}[1]{\nombre{#1}~\%}
\newcommand*{\prixSurface}[1]{\montant{#1}/m\up{2}}
\newcommand*{\PUP}{projet ubain partenarial\xspace}
\newcommand*{\sdp}{surface de plancher\xspace}

% Pour faciliter l'intégration denote de Magalie ou de Julie
\newcommand*{\CAD}{\cad}


\begin{document}

	\maketitle

	\chapter{Introduction générale au dorit de l'urbanisme et de l'aménagement}

	\section{Une police administrative spécialisée}
	
	\section{Un droit décentralisé}
	
		Ce droit a une dimension intercommunale.
	
	\section{Un droit instable}
	
		L'instabilité est due aux textes successifs et au contentieux local qui frappe les documents réglementaires.
	
	\section{Un droit multiple}
	
		Son caractère tient à la multiplicité des intervenant et s'exprime par des actes individuels et réglementaires.
	
	\section{La pyramide des normes}
	
		La documentation est thématique. Il y a une abscence de subsidiarité \dots
	% !TeX root = droitUrbanisme.tex

\chapter[Normes nationales et planif. locale]{Normes nationales et planification locale}

\section{Le règlement national d'urbanisme (RNU)}

	L'\articleDu[L]{101-3}{\cu} dispose que le règlement national d'urbanisme (RNU) a vocation à prévoir les règles générales applicables en matière d'utilisation des sols, à l'exception des collectivités d'outre-mer. Ces règles porteront notamment sur la localisation, la desserte, l'implantation et l'architecture des constructions, le mode de clôture et la tenue décente des propriétés foncières et des constructions.

	Ces règles générales, qui constituent le RNU, s'appliquent dans toutes les communes à l'exception des territoires dotés d'une carte communale, d'un plan d'occupation des sols (POS) rendu public ou d'un plan local d'urbanisme (PLU) approuvé ou d'un document en tenant lieu. Toutefois, le RNU peut désigner celles de ces règles qui demeurent applicables sur les territoires couverts par ces documents.

	\medbreak Il constitue à la fois la réglementation d'urbanisme minimale applicable sur tout le territoire en l'absence de règles plus précises et, pour certaines de ses dispositions, une réglementation nationale d'ordre public prévalant sur des règles locales qui ne respecteraient pas ces objectifs.

	Ce double caractère caractérise son champ d'application. Mais, du fait de la généralité de leur champ d'application, les dispositions du RNU ont un contenu généralement assez imprécis et un caractère principalement permissif.
	Elles laissent à l'autorité administrative chargée de les appliquer un large pouvoir d'appréciation. Ce qui donne une place importante au contrôle du juge sur la mise en œuvre du RNU.

	\medbreak La dernière refonte importante du RNU a été réalisée par ordonnance du \printdate{23/09/2015}, suivie de trois décrets en décembre 2015 et janvier 2016\footnote{\cu*, art. R.*\,111-1 s.}. Ce règlement est applicable à toutes les constructions, aménagements, installations et travaux faisant l'objet d'un permis de construire, d'un permis d'aménager ou d'une déclaration préalable ainsi qu'aux autres utilisations du sol régies par le code de l'urbanisme.

	\subsection{Contenu}

	Le RNU dans la rédaction qu'en donne le décret du \printdate{5/1/2007} est divisé en trois sections :
	\begin{itemize}
		\item la section 1 concernant la localisation et la desserte des constructions (\cu*, art. R.\,111-2 à R.\,111-15) ;
		\item la section 2 regroupant les règles relatives à l'implantation et au volume des constructions (\cu*, art. R.\,111-16 à R.\,111-20) ;
		\item et la section 3 étant relative à l'aspect des constructions (\cu*, art. R.\,111-21 à R.\,111-24).
	\end{itemize}

		\subsubsection{Règles relatives à la localisation et à la desserte des constructions}

			Le règlement national d'urbanisme contient tout d'abord des règles relatives à la localisation et à la desserte des constructions, aménagements, installations et travaux. Parmi celles-ci, est d'ordre public celle selon laquelle un projet peut être refusé ou n'être accepté que sous réserve de l'observation de prescriptions spéciales s'il est de nature à porter atteinte à la salubrité ou à la sécurité publique du fait de sa situation, de ses caractéristiques, de son importance ou de son implantation à proximité d'autres installations.

			S'appliqueront sur les territoires non couverts par un PLU les règles selon lesquelles un projet peut être refusé ou accepté sous réserves s'il est susceptible d'être exposé à des nuisances graves, dues notamment au bruit, celle selon laquelle un projet peut être refusé s'il n'est pas correctement desservi par des voies publiques ou privées ou encore la possibilité pour l'autorisation d'occupation du sol d'imposer notamment la réalisation de places de stationnement ou de voies privées nécessaires au respect des règles de desserte des constructions. De même, le RNU contient des règles relatives à la desserte des constructions par les réseaux d'eau potable et d'assainissement ou encore une règle permettant de refuser un projet qui implique la réalisation d'équipements publics hors de proportion avec les ressources de la commune d'implantation.

		\subsubsection{Règles relatives à l'implantation et au volume des constructions}

			Parmi les règles relatives à l'implantation et au volume des constructions, le RNU prévoit qu'une distance d'au moins trois mètres peut être imposée entre deux bâtiments non contigus situés sur un terrain appartenant à un même propriétaire ou encore qu'un bâtiment édifié en bordure de voie publique doit être distant de celle-ci d'au moins la hauteur de ce bâtiment, règles toutes deux applicables sur les territoires non couverts par un PLU. De même, à moins que le bâtiment à construire ne jouxte la limite parcellaire, le RNU prévoit que la distance comptée horizontalement de tout point de ce bâtiment au point de la limite parcellaire qui en est le plus rapproché doit être au moins égale à la moitié de la différence d'altitude entre ces deux points, sans pouvoir être inférieure à trois mètres.

			Le RNU prévoit que des dérogations à ces règles peuvent être accordées par décision motivée de l'autorité compétente en matière d'autorisations d'urbanisme (après avis du maire de la commune dans le cas où celui-ci n'est pas l'autorité compétente).

		\subsubsection{Règles relatives à l'aspect des constructions}

			Le RNU désigne comme d'ordre public la règle selon laquelle le projet peut être refusé ou accepté sous réserves si les constructions, par leur situation, leur architecture, leurs dimensions ou l'aspect extérieur des bâtiments ou ouvrages à édifier ou à modifier sont de nature à porter atteinte au caractère ou à l'intérêt des lieux avoisinants, aux sites, aux paysages naturels ou urbains ainsi qu'à la conservation des perspectives monumentales.

			Dans les secteurs déjà partiellement bâtis, présentant une unité d'aspect, l'autorisation de construire à une hauteur supérieure à la hauteur moyenne des constructions avoisinantes peut être refusée ou subordonnée à des prescriptions particulières.

			Ne s'applique pas, en revanche, sur les territoires couverts par un PLU, la règle selon laquelle les murs séparatifs et les murs aveugles apparentés d'un bâtiment doivent, lorsqu'ils ne sont pas construits avec le même matériau que les murs des façades principales, avoir un aspect qui s'harmonise avec celui des façades.

	\subsection{Effets}

\section[Les GPDU]{Les Grands Principes du Droit de l'Urbanisme (\articleCodifie*{L}{101-1} et \refArticle{L}{101-2})}

	Les principes et objectifs généraux figurent, en droit de l'urbanisme, aux \articlesCodifies{L}{101-1} à \refArticle{L}{ 101-3} du \cu.

	\subsection{Contenu}

		\paragraph{Dispositions de l'article L.\,101-1 du Code de l'urbanisme}

			Les collectivités publiques sont les gestionnaires et les garantes dans le cadre de leurs compétences du territoire français considéré comme le patrimoine commun de la nation.

			L'\articleDu[L]{101-1}{\cu} a pour objectif d'imposer aux différentes collectivités publiques l'harmonisation de leurs prévisions et de leurs décisions d'utilisation de l'espace en renvoyant à la réalisation des objectifs définis à l'\articleDu[L]{101-2}{\cu}.

			Le Conseil d'État a précisé qu'aucune disposition n'impose qu'un plan d'occupation des sols soit compatible avec ceux des communes voisines pour respecter le principe d'harmonisation posé par l'ancien \articleDu[L]{110}{\cu}\footnote{\jurisCE[109114]{24/6/1992}, Cne Pomerols}.

		\paragraph{Différents objectifs de l’article L.\,101-2 du Code de l'urbanisme}

			L'\articleDu[L]{101-2}{\cu} fait référence aux objectifs du développement durable tout en précisant que l'action des collectivités publiques en matière d'urbanisme vise à atteindre sept objectifs énoncés dans cet article. Ces différents objectifs n'ont cessé de s'allonger au fur et à mesure des modifications législatives.
			\begin{quote}
				\itshape
				1° L'équilibre entre :

				a) Les populations résidant dans les zones urbaines et rurales ;

				b) Le renouvellement urbain, le développement urbain maîtrisé, la restructuration des espaces urbanisés, la revitalisation des centres urbains et ruraux ;

				c) Une utilisation économe des espaces naturels, la préservation des espaces affectés aux activités agricoles et forestières et la protection des sites, des milieux et paysages naturels ;

				d) La sauvegarde des ensembles urbains et la protection, la conservation et la restauration du patrimoine culturel ;

				e) Les besoins en matière de mobilité ;

				2° La qualité urbaine, architecturale et paysagère, notamment des entrées de ville ;

				3° La diversité des fonctions urbaines et rurales et la mixité sociale dans l'habitat, en prévoyant des capacités de construction et de réhabilitation suffisantes pour la satisfaction, sans discrimination, des besoins présents et futurs de l'ensemble des modes d'habitat, d'activités économiques, touristiques, sportives, culturelles et d'intérêt général ainsi que d'équipements publics et d'équipement commercial, en tenant compte en particulier des objectifs de répartition géographiquement équilibrée entre emploi, habitat, commerces et services, d'amélioration des performances énergétiques, de développement des communications électroniques, de diminution des obligations de déplacements motorisés et de développement des transports alternatifs à l'usage individuel de l'automobile ;

				4° La sécurité et la salubrité publiques ;

				5° La prévention des risques naturels prévisibles, des risques miniers, des risques technologiques, des pollutions et des nuisances de toute nature ;

				6° La protection des milieux naturels et des paysages, la préservation de la qualité de l'air, de l'eau, du sol et du sous-sol, des ressources naturelles, de la biodiversité, des écosystèmes, des espaces verts ainsi que la création, la préservation et la remise en bon état des continuités écologiques ;

				7° La lutte contre le changement climatique et l'adaptation à ce changement, la réduction des émissions de gaz à effet de serre, l'économie des ressources fossiles, la maîtrise de l'énergie et la production énergétique à partir de sources renouvelables.
			\end{quote}

			Le principe d'équilibre est important. Le Conseil d'État a eu l'occasion d'annuler sur la base de l'erreur manifeste d'appréciation, la création d'une zone NAF (zone de construction future insuffisamment équipée qui peut être urbanisée) à vocation industrielle, artisanale et de services à proximité d'un parc boisé de 40 hectares, classé comme site à protéger, en considérant, notamment, que la révision du POS correspondante ne respectait pas le principe d'équilibre de l'\articleDu[L]{110}{\cu}\footnote{\jurisCE[115248]{21/10/1994}, Cne Bennwihr : JurisData \no 1994-052178 ; Lebon T., p. 1235 ; Dr. adm. 1994, comm. 695}. Le juge a eu l'occasion de vérifier le respect de l'équilibre entre l'extension urbaine, l'exercice des activités économiques et la préservation des milieux, sites et paysages naturels\footnote{\jurisCAA[09NC00452]{Nancy}{11/02/2010}, Cne Berentzwiller et a. : JurisData n° 2010-000719}. Un PLU a été annulé pour violation du principe d'équilibre entre le développement urbain et la gestion économe des espaces naturels et ruraux\footnote{\jurisCAA[10NT02174]{Nantes}{14/05/2012}, Cne Boissy-lès-Perche : JurisData \no 2012-014904}.

	\subsection{Effets}

\section{Les dispositions relatives à la loi montagne et au littoral}

	La loi montagne de 1985 codifiée en majeure partie par les \articlesDuEtSuivants[L]{122-1}{\cu} et littorale L121-1 et suivants
	Loi montagne a des dispositions protectrices du cadre pastorale agricole avec quasi impossibilité de densifier l’urbanisation.
	Ces lois s’imposent au PC, ou sont reprises dans le DTA ou SCOT et DDTU
	Sur la loi montagne il y a un processus les UTN Unité touristique nouvelle : possibilité dans une cogestion entre commune et intercommunalité et préfet, d’organiser une urbanisation importante dans les zones de montagnes.
	Ces UTN est un outil particulier prévu par un SCOT et un PLU et qui doit faire l’objet d’une autorisation UTN cogéré et co instruit par le préfet et la commune pour une urbanisation significative à usage d’habitation de commerce.
	Pour la loi littorale c’est plus complexe, il y a 3 types de zones. Essentiellement dans le CDU, elle s’applique à toutes les commune squi ont au moins une partie de façade au littoral ou un lac de + de 100 hectare
	Champ d’application matériel de la loi littoral : l’ensemble des travaux, quelqu’il soit soit et quel que soit le MO
	Il y a une gradation de contrainte de constructibilité réaménage par la loi ELAN :
	-	L121-16 sauf espace déjà urbanisé, aucune constructibilité dans la bande des 100m, interdiction quasi absolue, sauf deux types de travaux : certains travaux prévus par la loi comme station d’épuration et des voies de dessert, des voies publiques mais très encadrés. Deuxième exception, c’est lorsqu’il y a des activités qui exige la présence de l’eau (une thalasso n’exige pas la présence de l’eau) comme ostréiculture alors il faut justifie une présence obligatoire en limite de l’eau
	-	Deuxième espace entre la bande des 100m et 5-6 km par rapport au plus hautes eaux des rivages on est sur des espaces proches du rivage alors les textes considèrent que seule une urbanisation limitée peut être autorisée L121-13 du CDU qui renvoie à 2 notions la notion d’urbanisation limitée et espace proche du rivage
	Espace proche du rivage : apprécié objectivement par rapport au rivage. S’il y a une falaise, on ne prend pas en considération la falaise
	La covisibilité rivage – terrain, plus il y a une visibilité réciproque forte on peut considérer que l’on est encore dans un espace proche du rivage
	Urbanisation limitée : cela dépend des juridictions certains juges administratifs sont strictes, c’est une appréciation de fait. Le CE a considéré que urbanisation greffé à un port de plaisance arrêt PORTEGA position compréhensive, puis revirement le CE dit non le port de plaisance ne peut plus être l’alibi à a réalisation de constructibilité de complaisance
	Dernière notion, celle de l’extension de l’urbanisation en continuité avec des agglomération et village existant
	On a la bande de 100 m quasi constructibilité, puis espace proche du rivage, donc est ce que l’on peut imaginer une urbanisation plus lourde si extension en continuité par rapport à un village existant L121-8 du CDU seule manière de m’affranchir d’une urbanisation limitée sauf que dans ce cas le JA est exigent sur la notion de village ou agglomération existant : densité suffisamment fort d’urbanisation avec une centralité
	Le JA va regarder au cas par cas la constitution de l’urbanisation existante, densité de maison de voirie, d’équipement public suffisant pour considérer que l’on est en présence d’un village ou agglomération existant.

	L121-13 espace remarquable : espace sur commune littorale (comme parcelle en cravate) la loi littorale s’applique car la commune a une façade littorale, alors ce troisième espace dit remarquable s’appliquera.
	Les espaces remarquables contiennent des espèces animales et végétales propres aux espaces littoraux mais aussi des espaces avec des vestiges ou éléments patrimoniaux propres au littoral. Sur ces espaces, le JA pourra considérer que telle zonage d’un PLU au regard de  cette loi littorale ou tel PC ne sont pas compatibles avec ces espaces remarquables issus de la loi littorales.

	Erreur manifeste en méconnaissant des éléments remarquables qui auraient du être protégés, si EMA par le maire dans le PC alors illégalité alors annulation du PC.


\section{Les schémas directeurs et les nouveaux schémas de cohérences territoriale (SCOT)}

	\subsection{Contenu}

	\subsection{Effets (compatibilité)}

\section{Les Plans locaux d'urbanisme (PLU)}

	\subsection{Contenu des PLU (éléments obligatoires et facultatifs)}

		\subsubsection{Le rapport de présentation} (Articles R151-1 à R151-5)


		\subsubsection{Les orientations d'aménagement et de programmation}
		(Articles R151-6 à R151-8-1)

		\subsubsection{Le règlement}

		... ici reprise du cours ...

		\subsubsection{Les annexes}

		Elles forment la dernière partie du PLU. Elles sont opposables au même titre que les autre éléments du PLU.

		Les PPR sont annexé dans cette partie. Néanmoins, dès lors que ces PPR ne sont pas annexé, ils sont de manière autonome opposable, de la même manière que s'ils s'y trouvait.

	\subsection{Analyse détaillé d'un règlement de PLU}

		Le réglement de PLU est constituée de plusieurs parties décrites aux articles R 151-9 et suivants

		1. un index, ou lexique appelé souvent définition des élements de définition une meilleur compréhension. Ces définitions sont opposables tant au pétitionnaire qu'à l'administration.

		2. les règles générales. Les regles et disposition qui sont suceptible de s'appliquer sur plusieurs zones. Par exemple la reconstruction d'un bâtiment à l'identique. Parfois, on y trouve l'interdiction ... dans un article dédié
		R 151-21 sur la mutualisation des règles de gabarit qui vaut notamment sur les lotissement et les permis valant division. Ces deux catégories de permis particulier si la mutualisation n'a pas été interdite.

		Elle permet d'avoir une bonne combinaison des différentes règles

		3. Les zonages. C'est la partie la plus connue, celle qui permet de manière précise d elocaliser une parcelle et d'identifier rapidement les règles, ou << precriptions >>, qui vont s'appliquer. Ces zones sont réparties à partir des 4 grandes catégories par le curb :
			zone U
			AU
			A
			N
		La définition de ses zones au sein d'un PLU, que ce soit le réglement ou les documents graphiques, sont librement définies sous deux réserves :
			- ne pas faire d'erreur de droit. Par exemple faire apparaitre une incompatibilité avec un document supérieur (un SCOT, ou un zonage qui méconnaitréait des dispositions de la loi littorale
			- ne pas faire d'erreur manifeste d'appréciation (EMA), c'est à dire une erreur grossière relevée par le juge dans la définition du zonage. Par exemple, une zone U d'une commune qui serait limitrophe d'une zone industrielle d'une autre commune.

		\paragraph{Les zone U} R 151-18 (reprise de la définition littérale)

			Les capacité de desserte tant en réseaux qu'en voirie, est suffisante pour ...

			Elles peuvent faire l'objet d'une déclinaison par sous-secteur, et sous-secteur, permettant de mettre en place des règles différenciées.

		\paragraph{Les zone AU} R 151-20 (reprise de la définition littérale) ...

			Zone AU << mixte >> (alinéa 2) ... la constructibilité n'est possible (lotissement, PC valant division)

			Zone AU << stricte >>, (alinéa 3) ... c'est à dire qui ne permet pas en l'état la délivrance de permis. Il sera nécessaire de sortir de cette catégorie.

		\paragraph{Les zone A} R 151-22 (reprise de la définition littérale)

			On peut tout à fait des zones équipées.

			Il y a trois critères qui ne sont pas les seuls :
			- en raison de leur potentiel agronomique
			- ...

		\paragraph{Les zone N} R 151-18 (reprise de la définition littérale)

			équipé ou non

			5 critères ...

			Urbanisation très stricte mais pas impossible. lié à l'activité ou la destination de la zone. sont également admises des extensions de l'urbanisation

			Zone N d'urbanisation diffuse.

		\subsubsection{Les différents articles des zones}

			Jusqu'en 2016, et depuis plus de 40 ans, la nomenclature interne était prévue dans la partie A du code. Un nouve R 151-27 et suivants introduit une nouvelle structuration.

			Dans la pratique, les PLU restent structuré de cette manière.

			Il y avait 15 articles 2 ont été abrogé par la loi ALUR et un est apparu sur les économies d'éarge

			\paragraph{Articles 1 \& 2} Ceux sont les articles qui autorisent, interdisent, organisent ou restreignent les destinations des futurs bâtiments ou à rénover. C’est la vocation générale de la zone.

			Il peut y avoir des destination autorisée mais avec des contraintes.

			Par exemple ...

			R 151-27 et 28 destinations et sous destination

			SINASPIC

			\paragraph{Articles 3 \& 4} relatif à la desserte réseau et voirie. Limite ou non l’accès à la parcelle par sa largeur.

			\paragraph{Article 5} Il a été abrogé par la loi ALUR, à l'exception de 3 communes : Maison Laffite. Cet article encadrait la constructibilité en fonction de la taille des parcelles.

			\paragraph{Articles 6, 7 \& 8} Ce sont les règles de prospect (ou de recul). Ils sont considéré comme simple à expliquer mais difficiles à mettre en œuvre.

			L'article 6 traite des retraites par rapport à une voie public ou un espace public. Toute emprise créant. A priori l'ouverture au public suffit.

			L'article 7 traite du recul par rapport aux autres parcelles privées.

			L'article 8 est relatif aux règles de recul entre deux bâtiments sur une même parcelle. Le plus souvent cet article est plus clément que l'article 7. Les gestes architecturaux pour y échapper ont parfois été sévèrement.

			Les << conventions de cours commune >> permettent d'aménager ces règles en permettant d'appliquer les règles de l'article 8 alors même que ceux sont celles de l'article 7 qui auraient à s'appliquer.

			\paragraph{Article 9} C'est l'article relatif aux emprises qui permet de limiter la superficie du projet, y compris les emprises minérales.

			\paragraph{Article 10} C'est l'article relatif aux hauteurs des bâtiments. Cette hauteur peut être appréhendée au faitage ou à l'égout des toitures.

			Souvent il y a possibilité de dépasser cette hauteur pour un certain nombre d'édicule, si l'auteur du PLU l'a prévu et mentionné.

			...
			La règle que l'on voit communément est de prendre le point médian d'un segment parallèle à la voirie.

			Souvent le point bas se fait à partir du terrain naturel, qui est souvent repéré sur les plans du permis de construire ...

			\paragraph{Article 11} Article relatif aux aspects extérieurs des projets :

			Parfois se limite à une reprise de R 111-27, c'est à dire l'article du RNU relatif ...

			\paragraph{Article 12} relatif aux stationnements (auto, vélo, moto, poussette).  ...

			Les auteurs peuvent rentrer dans un relativement grand détail. Il est possible de : préciser la dimension des places, d'interdire ou autoriser les places commandées, ventiler le ratio des place par destination, \etc Depuis quelques années, la tendance est à la diminution des contraintes au sein de cet article pour diminuer l'importance des voitures en ville.

			possibilité .. par le paiement d'une redevance ... Cette participation n'est plus valable. Il n'est plus possible de s'exonérer de la réalisation des places. Cette disposition a été abrogée car les sommes perçues n'étaient pas utilisées pour réaliser les parking nécessaires.

			Il est possible de s'acquitter e l'obligation de réaliser les places, soit en les réalisant lui-même soit aller les rechercher dans un parking proche (sous forme d'une amodiation) ou, depuis la loi SRU, au sein d'un parking privé proche.

			Si CINASPIC, le plus souvent il n'y a aucun minimum pour les SINASPIC mais simplement la nécessité du montant ... nécessaire

			\paragraph{Article 13}  Article sur les espaces verts et le espace libres. C'est un article qui se décline souvent en plusieurs chapitre.

			Espace Boisé Classé pas forcément boisé ...

			Espace engazonnée. Les auteurs du PLU peuvent aussi exiger que la superficie soit en pleine terre

			Espace libre : par exemple des cours.

			\paragraph{Article 14} Article sur le COS. Il a été abrogé par la loi ALUR.

			\paragraph{Article 14} 3 mots

		\subsubsection{Effet } ...

		Les effets du PLU sont une opposabilité totale dans une relation de conformité, mais également deux regles :
			- regle complementaire, lorsqu'il y a un permis dans un perimetre d'oap, il doit y avoir compatibilité avec les orientations de l'oap
			- le legislateur que sur certain aspects, il y a lieu de pemettre des exceptions R 152-5 à 152-9

	\subsection{Procédure d'élaboration et de révision générale des PLU (les trois phases)}

		L'élaboration d'un PLU les 1ere en 2001 - 2002, suite à la loi SRU de décembre 200 qui a changé des ...

		Les 1er PLU ont été dans leur grande majorité attaqués. Les PLU ont supplanté les POS --- il n'y a plus de POS ou De PAZ applicable.

		Le PLU est l'outils de droit commun ...

		Qu'est ce qui pousse une commune à élaborer ou réviser son PLU. Depuis octobre 2017, il est fait obligation d'appliquer le RNU (R 111-1 et suivants) aux communes ayant toujours un POS. Les communes sont donc vivement incités à élaborer un PLU. Il y a certaine région, la Corse par exemple,  qui ont fait le choix de rester en RNU.

		La

		\paragraph{Réflexion sur le périmètre} Contrairement ax anciens POS, il n'est pas possible d'élborer un PLU partiel. Le PLU doit nécessairement couvrir l'intégralité du territoire communal --- ou intercommunal L153-1.

		Deux exceptions :
			- L 153-3, lorsqu'il y a fusion d'EPCI
			- plus fréquemment, lorsqu'il y a annulation partielle, ou déclaration d'illégalité partielle, d'un PLU.
			(12 mn environ) %L'annulation partielle peut provenir d'une décision d'un juge site à un recours sur un PC.

			Dans ce cas deux conséquences : le zonage ou l'article annulé ... on reprend le doc d'urbanisme antérieur.
			(zut j'ai décroché)

			La déclaration (exception )d'illégalité pas de possibilité de le conserver .. = paralysie

			Dans ces deux cas nous avons une disparition.

			Ce qui peut poser problème si des pétitionnaires << s'engouffrent dans la brèche >> et déposent des demande de PC. Il faut que la commune fasse vite.

			3 phaseq :
			(début partie 2)
			prescription
			arrert

		\paragraph{Première délibération}

		Article L 153-11 << cite
		Le Maire est incompétent pour lancer c'est le Conseil Municipal. On trouve dans la délib :
		 - objectifs poursuivis (protection, développement économique, intensification des liaisons douces, \etc)
		 - les modalités de la concertation. Elles sont librement définies par la collectivité. Fréquemment on trouve :
		 	- des réunions publiques
		 	-
		 	- un premier calendrier prévisionnel
		Il n'y a pas  de durée particulière. Ce qui est exigé est que la concertation s'arrête, et que le bilan soit tiré, avant l'enquête publique.

		Le 3\ieme{} alinéa de 143-11 prévoit la possibilité de << sursoir à statuer >> dès lors que des demandes d'autorisation seraient contraire aux prescriptions du futur PLU. En pratique :
			- Cette possibilité n'est possible que dans l'élaboration d'un PLU, ou une révision générale ;
			- Cette possibilité n'est possible qu'après la délibération qui consacre le débat sur les orientations générales du projet d'aménagement et de développement durable. modification en 2017 tenant compte de la jurisprudence pour ajouter --- auparavant il était possible de sursoir à statuer dès lors que la délibération ... l'élaboration du PLU ;

		\paragraph{Deuxième délibération sur le PADD.}

		Cela suppose qu'il y ait eu un débat, dont on conserve une trace.

		A partir de ce délibération, le sursis à statuer devient possible. Le juge reste souverain pour estimer si l à pris une consistance suffisante. La difficulté pour le pétitionnaire est qu'en dehors des moments où émerge ... et jusqu'à l'enquête publique, peu d'information filtre sur l'avancée des travaux d'élaboration. Le pétitionnaire est donc dans l'incertitude sur le niveau d'avancement. Plus le temps passe, plus il est probable que le sursis à statuer soit fondé.

		\paragraph{Arrêté de projet du PLU } Lorsque le document est

		Communication aux PPA L 153-16

		L 153-17 plus pernicieux car il évoque l'obligation de communiquer, à leur demande, à d'autres collectivités locales. Pour demander, il faut connaitre. Donc il faut informer pour pouvoir prouver qu'ils étaient au courant.

		Délais de 2 mois.

		Réception de l'ensemble des avis des PPA.

		Si il. La collectivité peut décider d'intégrer les observations.

		Si ces ajustements sont purmeent conformes. S'il y a des modifications substantielles -> nouvelles consultations des PPA. Ces avis ne sont pas publics à ce stade.

		\paragraph{Préparation de l'enquête publique}

		elle est important car elle va consacrer le PLU dans un état quasi définitif. Il ne pourra y avoir que des modifications mineures, pas de reise en cause de l'économie générale.

		Le calendrier ordinaire :

		Désignation du commissaire enquêteur. Le président du TA compétent est contacté par le maire pour désignation d'un commissaire enquêteur.
		A été jugé comme non impartial le mire adjoint d'une commune limitrophe.
		Le maire reçoit ourrier, puis

		Arrête avec le commissaire enquêteur les dates (au minimum 1 mois) et les modalités pratique ainsi que les dates de permanence du commissaire enquêteur. Parfois nomination d'une commission d'enquête.

		Ensuite information en deux temps. Plusieurs semaines dans la presse locale, en mairie, en mairie annexe, mentionnant l'ensemble des

		Code de l'environnement. Il y a eu énormément de contentieux avec annulation de PLU sur ce sujet.

		Modalité de l'enquête : accès libre et le plus large possible.
		Obligation d'un registre d'enquête public, qui doit obéir à des règles formelles très stricte.

		On trouve le PLU complet, l'ensemble des avis et les courriers d'envoie avec les dates de transmission, le porter à connaissance ...

		(1h22)

		.. les administrés de la commune, les administrés des communes voisines, les associations, les personnes morales de la communes, les autres collectivités locales \etc il n'y pas de limite. Il est possible de déposer des observations anonymes.

		Depuis quelques années, il y a en parallèle une consultation sur Internet.

		Fermeture des débats, par le commissionnaire enquêteur. Qui collationne . sauf s'il considère que l'enquête publique doit être prorogé. En cas de désaffection importante de la population, ou à l'inverse si l'enquête.

		(1:28)

		Le commissaire enquêteur est souverain de proroger ou non.

		Prépare 2 documents : le rapport et les conclusions.

		Dans le rapport, relate de manière fidèle toute l'enquête ... \etc Ne touche pas au fond du sujet.

		Pour la rédaction du rapport, en pratique est de rentrer dans le contenu des observations et de proposer à la collectivité une réunion de travail. Toute les questions

		(1:36) << une réponse personnalisée et circonstanciée >>

		%Si annulation du PLU alors responsabilité de la commune.

		Le calendrier de restitution : délai d'un mois, mais pas de sanction.

		Le commissaire peu rendre des conclusions favorables, sans réserves ni recommandations. C'est de plus en plus rare.
		Il peut émettre un avis favorable avec des recommandations.
		Il peut émettre un avis favorable avec des réserves ou des réserves exprès. L'avis ne vaut que si les réserves sont prises en considération. Sinon il faut considérer l'avis comme défavorable.
		Il peut émettre un avis défavorable.

		Juridiquement la commune peut passer outre les réserves. Mais cela renforce la possibilité de recours. Cela permet au Préfet de demander une suspension en urgence sans avoir à démontrer l'urgence.

		2 critères :
			- non substantiel
			- trouve leur source

		\paragraph{Délibération d'approbation}

		On considère que s'il y a un SCOT, il y a aucune raison pour que le PLU ne soit pas exécutoire


	\subsection{Les autres procédures d'adaptation des PLU}

		Il ne faut pas utiliser << modifcation >>
		 Du plus lourd au plus léger.

		 \subsubsection{Procédure de révision} L 153-31

		 \begin{quote}
		 	<< { \itshape
		 		Le plan local d'urbanisme est révisé lorsque l'établissement public de coopération intercommunale ou la commune décide :

		 		1\degres Soit de changer les orientations définies par le projet d'aménagement et de développement durables ;

		 		2\degres Soit de réduire un espace boisé classé, une zone agricole ou une zone naturelle et forestière ;

		 		3\degres Soit de réduire une protection édictée en raison des risques de nuisance, de la qualité des sites, des paysages ou des milieux naturels, ou d'une évolution de nature à induire de graves risques de nuisance.

		 		4\degres Soit d'ouvrir à l'urbanisation une zone à urbaniser qui, dans les neuf ans suivant sa création, n'a pas été ouverte à l'urbanisation ou n'a pas fait l'objet d'acquisitions foncières significatives de la part de la commune ou de l'établissement public de coopération intercommunale compétent, directement ou par l'intermédiaire d'un opérateur foncier.

		 		5\degres Soit de créer des orientations d'aménagement et de programmation de secteur d'aménagement valant création d'une zone d'aménagement concerté.
		 	} >>
		 \end{quote}

	 	Les 3 1\ieres{} hypothèses sont les plus importantes.

	 	Que dit le juge du 1. S'il y a changement d'une ou plusieurs orientation du PADD alors il s'agit d'une révision.

	 	Le 2 existait déjà du temps des POS. Dès lors que vous

	 	Le 3 est presque subjectif. C'est un PPRI

	 	\subsubsection{Procédure de modification} L 153-36

		 	\paragraph{La modification de droit commun} L 153-41

		 		\begin{quote}
		 			<< {\itshape
		 				Le projet de modification est soumis à enquête publique réalisée conformément au chapitre III du titre II du livre Ier du code de l'environnement par le président de l'établissement public de coopération intercommunale ou le maire lorsqu'il a pour effet :

		 				1\degres Soit de majorer de plus de 20 \% les possibilités de construction résultant, dans une zone, de l'application de l'ensemble des règles du plan ;

		 				2\degres Soit de diminuer ces possibilités de construire ;

		 				3\degres Soit de réduire la surface d'une zone urbaine ou à urbaniser ;

		 				4\degres Soit d'appliquer l'article L. 131-9 du présent code.
		 			} >>
		 		\end{quote}

	 			Les 3 1\iers{} cas sont les plus importants.
	 			\begin{enumerate}
	 				\item Le mot zone n'est pas limité à une mais peut concerner un secteur ou un sous-secteur. Il faut apprécier si la modification des règles gabaritaires\footnote{articles 6, 7, 8, 9, 10 et 13 du règlement}

	 				\item  valable pour toute les zones, même en zone N

	 				\item Si on réduit la surface .
	 			\end{enumerate}

	 		Souvent 6 ou 7 mois

	 		\paragraph{La modification simplifiée} L 153-45

		 		\begin{quote}
		 			<< {\itshape
		 				La modification peut être effectuée selon une procédure simplifiée :

		 				1\degres Dans les cas autres que ceux mentionnés à l'article L. 153-41 ;

		 				2\degres Dans les cas de majoration des droits à construire prévus à l'article L. 151-28 ;

		 				3\degres Dans le cas où elle a uniquement pour objet la rectification d'une erreur matérielle.

		 				Cette procédure peut être à l'initiative soit du président de l'établissement public de coopération intercommunale ou du maire d'une commune membre de cet établissement public si la modification ne concerne que le territoire de cette commune, soit du maire dans les autres cas.
		 			} >>
		 		\end{quote}

			 	L'erreur matérielle. La plus grave et la plus courante est l'erreur du fonds de plan.

			 	Le cas principal est le 2\ieme{}. Les effets sont importants, tant opérationnellement qu'en terme de contentieux. L  L 153-47 prévoit que les documents << sont mis à disposition du public pendant un mois, dans des conditions lui permettant de formuler ses observations >> sans le formalisme de l'enquête publique.

			 	Cependant prête le flan à la critique du détournement de pouvoir si la modification est trop ciblée, ou, à l'inverse, si la modification est trop large à vous rapprocher du cas de la modification de droit commun.

			 	On observe souvent un (24:30)

			 	Souvent 3 ou 4 mois pour faire une modification simplifiée.

			 	Très intéressant pour tout ce qui ne rentre pas dans le champ de la modification classique. Ex. : réduire les ratios de stationnement.

		 	\subsubsection{Procédure de mise en compatibilité} L 153-49

		 		\paragraph{Déclaration de projet}

		 		Le véritable intérêt résulte de la pratique. C'est l'adaptation à un projet.

		 		Le temps d'élaboration est de 6 à 7 mois

		 		Très souvent consiste en un secteur de plan masse\footnote{
		 			article R 151-40 « Dans les zones U, AU,
		 			dans les secteurs de taille et de capacité d'accueil limitées délimités en application de
		 			l'article L. 151-13, ainsi que dans les zones où un transfert des possibilités de construction a
		 			été décidé en application de l'article L. 151-25, le règlement peut définir des secteurs de plan
		 			masse côté en trois dimensions »

			 		Le plan de masse fixe les volumes constructibles en déterminant a minima les règles
			 		d’implantation, d’emprise et de hauteur. Il doit faire apparaître avec suffisamment de
			 		précision l’implantation des constructions projetées
			 		, sans avoir pour autant à préciser leur
			 		localisation exacte.
		 		}.

	\subsection{Les certificats d'Urbanise (CU)}

\section{Les cartes communales}

\section{L'impact du droit de l'environnement}

	\subsection{L'enquête publique (PLU et certains PC)}

	\subsection{Étude d'impact et évaluation environnementale}

	\subsection{Les mesures favorables aux énergies renouvelables}

	% !TeX root = droitUrbanisme.tex
\chapter{L'aménagement}

L 300-1 \& L 300-2

	\begin{quote}
		\textbf{\articleCodifie{L}{300-1}}

		<< {\itshape Les actions ou opérations d'aménagement ont pour objets de mettre en œuvre un projet urbain, une politique locale de l'habitat, d'organiser le maintien, l'extension ou l'accueil des activités économiques, de favoriser le développement des loisirs et du tourisme, de réaliser des équipements collectifs ou des locaux de recherche ou d'enseignement supérieur, de lutter contre l'insalubrité et l'habitat indigne ou dangereux, de permettre le renouvellement urbain, de sauvegarder ou de mettre en valeur le patrimoine bâti ou non bâti et les espaces naturels.}

		\lips >>
	\end{quote}

	ARticle essentiel qui permet la qualification et la justification

	Préemption implique ....

	Très nombreux exemples tirés de la jurisprudence.
	\begin{itemize}
		\item création d'un pôle
		\item
	\end{itemize}

	\begin{quote}
		\textbf{\articleCodifie{L}{300-2}}

		<< {\itshape } Les projets de travaux ou d'aménagements soumis à permis de construire ou à permis d'aménager, autres que ceux mentionnés au 3° de l'article L. 103-2, situés sur un territoire couvert par un schéma de cohérence territoriale, par un plan local d'urbanisme ou par un document d'urbanisme en tenant lieu ou par une carte communale peuvent faire l'objet de la concertation prévue à l'article L. 103-2. Celle-ci est réalisée préalablement au dépôt de la demande de permis, à l'initiative de l'autorité compétente pour statuer sur la demande de permis ou, avec l'accord de celle-ci, à l'initiative du maître d'ouvrage.>>
	\end{quote}

	Concertation (cf. l)

	\articleCodifie{L}{300-4} a été introduit comme étant un mandat. Prénete peu d'intert compte tenu de la loi MOP.

	L\articlesCodifies{L}{300-5} est suivants sont fondamentaux. Ils sont relatifs aux concessions d'aménagements.

	Loi de 2005 vient mettre le droit français en conformité avec le droit européen relatif à la concurrence.

	\begin{quote}
		\textbf{\articleCodifie{L}{300-2}}

		<< {\itshape } L'État et les collectivités territoriales, ainsi que leurs établissements publics, peuvent concéder la réalisation des opérations d'aménagement prévues par le présent code à toute personne y ayant vocation.

		\medskip L'attribution des concessions d'aménagement est soumise par le concédant à une procédure de publicité permettant la présentation de plusieurs offres concurrentes, dans des conditions prévues par décret en Conseil d'Etat. \lips

		\medskip Le concessionnaire assure la maîtrise d'ouvrage des travaux, bâtiments et équipements concourant à l'opération prévus dans la concession, ainsi que la réalisation des études et de toutes missions nécessaires à leur exécution. Il peut être chargé par le concédant d'acquérir des biens nécessaires à la réalisation de l'opération, y compris, le cas échéant, par la voie d'expropriation ou de préemption. Il procède à la vente, à la location ou à la concession des biens immobiliers situés à l'intérieur du périmètre de la concession.  >>
	\end{quote}

	Les opérations peuvent être faite << en régie >>, ou via des concessionnaires ou des opérateurs qui peuvent être de deux types :
	\begin{itemize}
		\item SEM << \emph{in house} >> ou << quasi régie >>
		\item Société privé
	\end{itemize}
	Pas de concession sans mise en concurrence, à l'exception du cas des concessionnaires \emph{in house}.

	Le concessionnaire assure , il en est raisonnable. Il assure en amont les étude. Et il a une mission particulière grâce à l'expro : il peut .

	Le \cu organise très complètement le rôle.

	\begin{itemize}
		\item \articlesCodifiesS{R}{300-4} concession pour lesquels le risque financier prépondérant
		\item \articlesCodifies{R}{300-11} pas de transfert de risque financier prépondérant sur la tête du concessionnaire
	\end{itemize}
	Second cas : subvention, prêt public, \etc Situation proche d'un marché public.

	Pas de concession si pas d'équipement public.

\section{Les droits de préemption et de priorité}

	Vu avec Me Perinet Marquet

\section{Actions et opération d'aménagement}

	\subsection{Les concessions}

		\paragraph{1\ier{} cas : la concession en régie.}

		Intérêt si les terrains sont déjà maitrisé par la commune et qu'elle possède les compétences suffisantes lui permettant d'assumer l'execution.

		Représente environ \pourcent{5} des concessions

		\paragraph{2\ieme{} cas : la concession en régie.}

		...

		Risques techniques
		risque commercial

	\subsection{Les PPA et les GOU}

\section{Les Zones d'aménagement concerté (ZAC)}

	Déroulé chronologique :

	Pas de ZC si pas de nécessite de réaliser des équipements publics.
	Concertation. La commune n'est pas lié par la concertation.

	3 solutions :
	\begin{itemize}
		\item pas plus loin
		\item
		\item
	\end{itemize}

	C'est l'approbation du dossier de création de la ZAC a
	\begin{enumerate}
		\item Plan d'emprise
		\item
		\item Option fiscale : le plus souvent participation ... et plus la TA
		\item PGC prévisionnel
		\item l'étude d'impact document très volumineux
	\end{enumerate}

	Création de la ZAC, effets juridiques :
	\begin{enumerate}
		\item changement de mode de financement
		\item droit de préemption ou d'expropriation peut s'appliquer dans tous le périmètre de la ZAC
		\item possibilité de mettre en demeure la collectivité locale concédante d'acquérir (droit de délaissement)
		\item plus de nécessité d'autorisation administrative préalable pour les découpages fonciers pour l'aménageur
	\end{enumerate}

	\subsection{Principes généraux}

		\subsubsection{Autonomie relative avec le PLU}

		Il y avait avec la loi SRU, tout lien a été rompu. depuis 2000
		Deux exceptions
		\begin{itemize}
			\item les principales localisation ans un PLU peut apparaitre
			\item depuis la loi ELAN, approbation d'un OAP peut valoir création de ZC
		\end{itemize}

		\subsubsection{Successions d'acte}

	\subsection{Concertation préalables et bilan}

	\subsection{Dossier de création}

		\subsubsection{Mode de réalisation}

		\subsubsection{Option fiscale}

			\paragraph{Participation}

			\paragraph{Taxe d'aménagement}

		\subsubsection{Étude d'impact}

			Fondement 1976. N'a jamais cessé d'augmenter quant à son contenu.

			Critique régulière de l'insuffisance des études d'impact. Raison de l'annulation de la ZAC du << Triangle de Gonesse >>

	\subsection{Désignation de l'aménageur}

		introduit par la loi de 2005

		Phase très chronophage : 3 à 4 mois

		Délibération qui lance la consultation

		Publicité

		Réponse

		Critères non exclusifs :
		\begin{enumerate}
			\item suffisance financière
			\item expertise technique
		\end{enumerate}
		et
		\begin{enumerate}
			\item meilleure maitrise foncière, abandonné depuis
			\item emploi local
		\end{enumerate}

		Contestation possible par référé pré-contractuel. \lips

		\subsubsection{Fin de la désignation discrétionnaire de l'aménageur}

		\subsubsection{Consultation obligatoire : lois de juillet 2005 et 2009}

		\subsubsection{Ordonnance de janvier 2016 sur les concessions d'aménagement}

	\subsection{Dossier de réalisation}

		\begin{enumerate}
			\item Programme d'équipement public
			\item Modalités prévisionnelles de financement
			\item Programme des équipements publics
		\end{enumerate}

		\begin{quote}
			\articleCU{R}{311-7}
			<< {\itshape La personne publique qui a pris l'initiative de la création de la zone constitue un dossier de réalisation approuvé, sauf lorsqu'il s'agit de l'État, par son organe délibérant. Le dossier de réalisation comprend :

				a) Le projet de programme des équipements publics à réaliser dans la zone ; lorsque celui-ci comporte des équipements dont la maîtrise d'ouvrage et le financement incombent normalement à d'autres collectivités ou établissements publics, le dossier doit comprendre les pièces faisant état de l'accord de ces personnes publiques sur le principe de la réalisation de ces équipements, les modalités de leur incorporation dans leur patrimoine et, le cas échéant, sur leur participation au financement ;

				\lips

%				b) Le projet de programme global des constructions à réaliser dans la zone ;
%
%				c) Les modalités prévisionnelles de financement de l'opération d'aménagement, échelonnées dans le temps.
%
%				Le dossier de réalisation complète en tant que de besoin le contenu de l'étude d'impact mentionnée à l'article R. 311-2 ou le cas échéant la ou les parties de l'évaluation environnementale du plan local d'urbanisme portant sur le projet de zone d'aménagement concerté, conformément au III de l'article L. 122-1-1 du code de l'environnement notamment en ce qui concerne les éléments qui ne pouvaient être connus au moment de la constitution du dossier de création.
%
%				L'étude d'impact mentionnée à l'article R. 311-2 ou le cas échéant la ou les parties de l'évaluation environnementale du plan local d'urbanisme portant sur le projet de zone d'aménagement concerté ainsi que les compléments éventuels prévus à l'alinéa précédent sont joints au dossier de toute enquête publique ou de toute mise à disposition du public concernant l'opération d'aménagement réalisée dans la zone.
			} >>
		\end{quote}

		Il ne peut être mis à la charge de l'aménageur ...

		Ce tableau peut vivre, mais à chaque fois il y a nécessité de modifier le dossier. Il y a donc nouvelle délibération pour entériner le dossier modifié.



		\subsubsection{Modalités prévisionnelles de financement}

			Le deuxième tableau renvoie aux modalités de financement.

			La quasi totalité des recettes proviennent de la vente.

			Les charges sont constituées :
			\begin{itemize}
				\item en premier lieu par les acquisitions foncières.
				Les aménageurs utilisent souvent la technique de << l'escargot >>, consistant à acquérir et commercialiser les terrains l'un après l'autre,de manière à éviter d'acheter beaucoup de terrains.

				\item Les coût de réalisation des équipements publics.

				\item La viabilisation des terrains

				\item Garanties financières

				\item Frais de structure

				\item Frais commerciaux.

				\item Frais divers. Y compris le contentieux.
			\end{itemize}

		\begin{quote}
			\articleCU{L}{311-4}
			<< \lips

			{\itshape Lorsqu'une construction est édifiée sur un terrain n'ayant pas fait l'objet d'une cession, location ou concession d'usage consentie par l'aménageur de la zone, une convention conclue entre la commune ou l'établissement public de coopération intercommunale compétent pour créer la zone d'aménagement concerté et le constructeur, signée par l'aménageur, précise les conditions dans lesquelles le constructeur participe au coût d'équipement de la zone. La convention constitue une pièce obligatoire du dossier de permis de construire ou de lotir.

				\medskip La participation aux coûts d'équipement de la zone peut être versée directement à l'aménageur ou à la personne publique qui a pris l'initiative de la création de la zone si la convention conclue avec le constructeur le prévoit.}

			\lips >>
		\end{quote}
		Que ce soit le propriétaire ou le constructeur qui va valoriser le terrain, qui doit mettre en place une convention de participation avec la ville.

		Il n'y a pas de quantum imposé. La pratique est de faire une règle de 3 le ... rapporté au coûts des équipements publics.

		Convention avec la collectivité locale, il faut donc une délibération du CM habilitant le signataire.

		Elle contient le montant, l'échéancier de paiement.


		\subsubsection{Programme d'équipement public}

		\subsubsection{Programme global de construction}

	\subsection{Commercialisation des terrains par l'aménageur}

		\begin{quote}
			\textbf{\articleCU{L}{311-4}} :
			<<
			{\itshape Les cessions ou concessions d'usage de terrains à l'intérieur des zones d'aménagement concerté font l'objet d'un cahier des charges qui indique le nombre de mètres carrés de surface de plancher dont la construction est autorisée sur la parcelle cédée. Le cahier des charges peut en outre fixer des prescriptions techniques, urbanistiques et architecturales imposées pour la durée de la réalisation de la zone.

			\medskip Le maire ou le président de l'établissement public de coopération intercommunale, dans les cas où la création de la zone relève de la compétence du conseil municipal ou de l'organe délibérant de l'établissement public de coopération intercommunale, ou le représentant de l'Etat dans le département dans les autres cas, peut approuver le cahier des charges. Si le cahier des charges a été approuvé, et après qu'il a fait l'objet de mesures de publicité définies par décret, celles de ses dispositions qui sont mentionnées au premier alinéa sont opposables aux demandes d'autorisation d'urbanisme.

			\medskip Le cahier des charges devient caduc à la date de la suppression de la zone. Les dispositions du présent alinéa ne sont pas applicables aux cahiers des charges signés avant l'entrée en vigueur de la loi \no 2000-1208 du 13 décembre 2000 relative à la solidarité et au renouvellement urbains.} >>
		\end{quote}

		Il faut en retenir qu'il y a deux cahiers des charges possibles :
		\begin{itemize}
			\item le premier, obligatoire, indique l'enveloppe de constructibilité cédée en terme de \sdp et ne constitue pas réellement un cahier des charges ;
			\item le second, facultatif, permet de préciser tout ce qui n'est pas prévu à l'article 11 du PLU, les prescriptions techniques, urbanistiques et architecturales.
		\end{itemize}

	On peut imaginer qu'il y ait des clauses qui sont étrangères à la question urbaines, mais dans ce cas elles n'ont d'effet que contractucel. Mais le cahier des charges ne peut être contraire au PLU. Il peut être plus contraignant, mais pas contraire. dans ce cas, le PLU prime.

	\paragraph{Modalité de validation} Relativement simple. Dans la majorité des cas un arrêté municipal approuve le cahier des charges. Pas de nécessité de passer en \CM

	\paragraph{Effet juridique} Double effet : contractuel et réglementaire.

	\begin{enumerate}
		\item \textbf{Effet contractuel.} La question se pose pour le primo acquéreur. Elle se pose également pour les acquéreurs successifs. On peut se

		Si la ZAC est supprimée, la question ne se pose plus (enfin plutôt).

		Peu de jurisprudence.

		\item \textbf{effet réglementaire} Mêmes effets que les prescriptions d'un PLU. L'organisme instructeur doit exercer un double contrôle : d'une part le PLU, d'autre part le cahier des charges.
	\end{enumerate}



		\subsubsection{Cahier des charges obligatoire}

		Nullité relative en cas d'abscence

		\subsubsection{Surface de plancher par parcelle}

		\subsubsection{Cahier des charges facultatifs}

		\subsubsection{Prescription techniques et architecturales}

Suppression de ZAc

\begin{quote}
	\textbf{\articleCU{L}{311-412}} :
	<<
	{\itshape La suppression d'une zone d'aménagement concerté est prononcée, sur proposition ou après avis de la personne publique qui pris l'initiative de sa création, par l'autorité compétente, en application de l'article L. 311-1, pour créer la zone. La proposition comprend un rapport de présentation qui expose les motifs de la suppression.

	\medskip La modification d'une zone d'aménagement concerté est prononcée dans les formes prescrites pour la création de la zone.

	\medskip La décision qui supprime la zone ou qui modifie son acte de création fait l'objet des mesures de publicité et d'information édictées par l'article R. 311-5.} >>
\end{quote}

Une modification substantielle, tant sur les aspects programmatiques ou financiers \lips

Lorsque la ZAC est achevée, soit car le programme des équipements publics est achevé ou l'ensemble des lots cédés, ou les deux ensembles (idéalement).

	\chapter{Fiscalité de l'urbanisme}
	\chapter{Permis de construire}

\section{Caractères généraux}

\section{Contenu du dossier de demande et pièces complémentaires}

\section{Déroulement et incidents dans la phase d'instruction}

\section{Résultats de l'instruction}

\section{Les différents type de permis de construire}

	\subsection{PC et PCM}
	
	\subsection{PC rectificatif}
	
	\subsection{PC de régularisation}
	
	\subsection{PC valant division (article R.~431-24)}
	
	\subsection{PC précaire}
	
\section{Transfert de PC}

	\subsection{Transfert de droit commun et total}
	
	\subsection{Transfert partiel}
	
\section{Prorogation du PC}
	
	\subsection{Prorogation sans démarrage du chantier}
	
	\subsection{Prorogation du fait des travaux engagés}
	
\section{Retrait du PC}
	
	\subsection{Principe et modalités du retrait}
	
	\subsection{Retrait à tout moment dans l'hypothèse d'un PC obtenu grâce à des manœuvres commises par le pétitionnaire}
	
\section{Le régime de la conformité administrative (totale ou partielle)}
	\chapter{Lotissements : déclaration préalable et permis d'aménager}
	\chapter[Contentieux]{Les contentieux de l'urbanisme et de l'aménagement}

\section{Situations litigieuses : contentieux ou protocole transactionnel ?}

\section{Les contentieux administratifs}

	\subsection{Le contentieux de la légalité}
		
		\subsubsection{Gestion de l'affichage et des délais}
		
		\subsubsection{Les moyens de recevabilité des actions des requérants}
		
		\subsubsection{Moyens juridiques}
		
		\subsubsection{Pratique du référé-suspension}
		
		\subsubsection{Pratique de l'annulation partielle et de la régularisation}
		
	\subsection{Le contentieux indemnitaire}
	
		\subsubsection{Les 3 fondements de la responsabilité}
		
		\subsubsection{Les différents préjudices indemnisables}
		
		\subsubsection{Les causes exonératoires}
		
		\subsubsection{Le processus de mise en œuvre de la responsabilité de l'Administration}
		
\section{Le contentieux pénal}

\section{Le contentieux civil}

	\tableofcontents

\end{document}
